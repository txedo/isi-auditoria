\begin{spacing}{1.4}

La evaluaci�n y gesti�n de riesgos es un tema que no est� abordado de
una forma eficiente en la organizaci�n ya que, aunque la empresa
dispone de un plan de riesgos, �ste ha sido elaborado por algunos de
los miembros del Consejo de Administraci�n y no existe ning�n informe
en el cual se indique qui�nes de estos miembros estaban presentes en la
elaboraci�n del plan ni las personas a las que consultaron a la hora
de realizarlo. Adem�s, tampoco existe ning�n registro de la t�cnica de
identificaci�n de riesgos que emplearon para seleccionarlos, as� como el 
c�lculo de la probabilidad de ocurrencia y el impacto, lo cual puede 
significar que la elecci�n de �stos par�metros
fuese arbitraria o manipulada. En el Anexo ?? se muestra el acta de
los asistentes a dicha reuni�n. 

Un aspecto a destacar en los riesgos seleccionados y planificados es
que se ci�en al �mbito software dej�ndo de lado el �mbito hardware, no
teniendo en  cuenta, por ejemplo, factores externos sobre los cuales
la organizaci�n puede no tener ning�n control debido a su naturaleza
(terremotos, colisiones de aviones, etc...) o incluso el robo de material o
incendios. En el Anexo ?? se puede observar una lista de los riesgos
que deber�an ser revisados y/o considerados.

Adem�s, el CIO y el Consejo de Administraci�n han ido elaborando una
lista de los nuevos riesgos que han ido azotando a la organizaci�n
pero no los han documentado en t�rminos del impacto que �stos
produjeron ni las estrategias de mitigaci�n o planes de contigencia
que se llevaron a cabo para combatirlos con el objetivo de reutilizarlos
o de mejorarlos. Debido a �sto, algunos riesgos se han producido
en varias ocasiones y no se ha ido reduciendo el impacto. Como prueba
de ello, en el Anexo ?? se puede apreciar que la empresa ha sufrido
dos apagones de luz generales en seis meses, las p�rdidas econ�micas
que supusieron �mbos, el coste de haber desarrollado un plan de
contigencia, y las p�rdidas que se habr�an producido si dicho plan se
hubiese llevado a cabo.

\end{spacing}