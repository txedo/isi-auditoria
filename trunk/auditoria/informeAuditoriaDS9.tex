Administrar la configuraci�n

La empresa dispone de una herramienta de soporte y un repositorio central para 
almacenar informaci�n relevante sobre los elementos de configuraci�n pero no se 
encuentra actualizado con las nuevas necesidades y adquisiciones de la empresa, 
es decir, inicialmente se grabaron todos los activos pero no se han ido grabando 
los cambios que �stos han sufrido. Un claro ejemplo de esta falta es la existencia 
de licencias caducadas de aplicaciones y que a�n se est�n instalando en nuevas 
computadoras. Este descontrol ha llevado a la empresa a re-adquirir 
licencias que ya pose�a y gastar un dinero innecesario (ver Anexo ??).

Por otro lado, no existe un prodecimiento que defina qu� software debe ser instalado 
en una m�quina cuando es introducida en la empresa. Esta labor la llevan a cabo 
dos t�cnicos que van instalando aplicaciones en los ordenadores conforme los usuarios 
las van necesitando y sin crear ning�n tipo de informe sobre los cambios que van 
haciendo a cada computadora.

Adem�s, se ha encontrado software personal y/o no licenciado (ver Anexo ?? -lista
de programas-) instalado en las computadoras de los empleados. Se le ha pedido 
al responsable en cuesti�n los informes de las revisiones pertinentes que se han 
llevado a cabo en dicho �mbito y no hemos obtenido respuesta alguna.

