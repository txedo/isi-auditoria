La evaluación y gestión de riesgos es un tema que no está abordado de
una forma eficiente en la organización ya que, aunque la empresa
dispone de un plan de riesgos, éste ha sido elaborado por algunos de
los miembros del Consejo de Administración y no existe ningún informe
en el cual se indique quiénes de estos miembros estaban presentes en la
elaboración del plan ni las personas a las que consultaron a la hora
de realizarlo. Además, tampoco existe ningún registro de la técnica de
identificación de riesgos que emplearon (p.e. Delphi) para
seleccionarlos, lo cual puede significar que la elección de éstos
fuese arbitraria o manipulada. En el Anexo ?? se muestra el acta de
los asistentes a dicha reunión. 

Un aspecto a destacar en los riesgos seleccionados y planificados es
que se ciñen al ámbito software dejándo de lado el ámbito hardware, no
teniendo en  cuenta, por ejemplo, factores externos sobre los cuales
la organización puede no tener ningún control debido a su naturaleza
(terremotos, colisiones de aviones, etc...) o incluso el robo de material o
incendios. En el Anexo ?? se puede observar una lista de los riesgos
que deberían ser revisados y/o considerados.

Además, el CIO y el Consejo de Administración han ido elaborando una
lista de los nuevos riesgos que han ido azotando a la organización
pero no los han documentado en términos del impacto que éstos
produjeron ni las estrategias de mitigación o planes de contigencia
que se llevaron a cabo para combatirlos con el objetivo de reutilizarlos
o de mejorarlos. Debido a ésto, algunos riesgos se han producido
en varias ocasiones y no se ha ido reduciendo el impacto. Como prueba
de ello, en el Anexo ?? se puede apreciar que la empresa ha sufrido
dos apagones de luz generales en seis meses, las pérdidas económicas
que supusieron ámbos, el coste de haber desarrollado un plan de
contigencia, y las pérdidas que se habrían producido si dicho plan se
hubiese llevado a cabo.

