% -*- coding: utf-8 -*-

\subsection{Problemas relacionados con los cambios}

\begin{spacing}{1.4}

La empresa carece de procediminetos de administraci�n de cambios formales para manejar de manera est�ndar todas las solicitudes que afectan a:
\begin{enumerate}
	 \item Cambios de aplicaciones software.
	 \item Cambios de procedimientos.
	 \item Cambios de procesos.
	 \item Cambios de par�metros del sistema.
	 \item Cambios de par�metros de servicio.
	 \item Cambios de las plataformas fundamentales.
\end{enumerate}

Del mismo modo, los cambios realizados no son registrados en ning�n sitio ni previamente autorizados. Adem�s, no se garantizan los resultados de los cambios antes de realizarse
ya que no existe ning�n metodo de evaluaci�n de los impactos de los cambios en el sistema operacional y su funcionalidad. 

La empresa tambi�n carece de un plan de cambios de emergencia que defina, plantee, evalue y autorice los posibles cambios de emergencia.

Con respecto a los cambios aplicados no existe un sistema de seguimiento y reporte de los mismos que mantengan informados a los solicitantes del cambio y a los interesados
relevantes, acerca del estado de cambio de las aplicaciones, procedimientos, procesos, par�metros del sistema y del servicio y de las plataformas fundamentales.

En consecuencia con todo lo anterior, se carece de un proceso de revisi�n que garantice la implantaci�n corrrecta y completa de los cambios.


\end{spacing}