\subsection {Gesti�n de la configuraci�n y control de cambios}

\begin{spacing}{1.4}

La empresa dispone de una herramienta de soporte y un repositorio central para almacenar informaci�n relevante sobre los elementos de configuraci�n hardware y software. El objetivo es garantizar la integridad de las configuraciones, pero este repositorio no se encuentra actualizado con las nuevas adquisiciones y necesidades de la empresa, ni con la evoluci�n de las TI.

En el Anexo 6 se observa un fragmento un documento, proporcionado por el Gerente de configuraci�n, que especifica la configuraci�n inicial de los equipos que utilizar�n los empleados. Como se puede apreciar, el software que inicialmente se instalar� en estos equipos est� obsoleto y no existe una norma est�ndarice la instalaci�n de parches y actualizaciones de seguridad, as� como la configuraci�n de servicios y par�metros del sistema. Esto hace que los sistemas no sean seguros y sean cada vez m�s vulnerables a ataques inform�ticos (ver Anexo 7, proporcionado por el Jefe de Seguridad).

Otra de las consecuencias que acarrea no tener actualizado el repositorio afecta al uso y adquisici�n de licencias. En el Anexo 8 se puede apreciar que en el �ltimo a�o se han adquirido licencias duplicadas para el antivirus y que a�n as� existen equipos con licencias caducadas (ver Anexo 9). Este problema, cuantificando �nicamenten el caso del antivirus, ha supuesto un total de 676 euros adicionales en 6 meses, un gasto totalmente innecesario.

Por otro lado, los cambios realizados no son registrados en ning�n documento ni previamente autorizados por el Director de Inform�tica, y no se mantiene una l�nea base de los elementos de la configuraci�n para todos los sistemas y servicios como punto de comprobaci�n al que volver tras el cambio. Adem�s, no se garantizan los resultados de los cambios antes de realizarse ya que �stos no est�n sujetos a ning�n tipo de norma y cada usuario los hace libremente. Como prueba de ello, en el Anexo 10 se observan capturas de pantalla de software personal, no licenciado o no alineado con el negocio de la empresa y que ha sido instalado en los equipos. Esto podr�a haberse corregido si los equipos se hubiesen sido revisados peri�dicamente por el Gerente de configuraci�n.

\end{spacing}