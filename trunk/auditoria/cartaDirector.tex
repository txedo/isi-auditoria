

\begin{flushright}
	\textbf{\today} \\
\end{flushright}

Estimado Sr. Director General:

Tras realizar la auditor�a a la sede de su empresa DIGSOL en Toledo,
 tal y como nos solicit�, le resumimos los problemas m�s
 graves que se han encontrado:

\begin{milista}
	\item \textbf{Ausencia de un plan estrat�gico de Tecnolog�as de la Informaci�n:} no existe una integraci�n total de las Tecnolog�as de la Informaci�n en la empresa, lo que conlleva que no se aprovechen correctamente las tecnolog�as, no se investigue en ellas para obtener mejores beneficios para la empresa, se realicen inversiones innecesarias y se prioricen incorrectamente los proyectos inform�ticos.
	\item \textbf{Problemas en la gesti�n de la configuraci�n y control de cambios:} los documentos que definen la configuraci�n inicial de los equipos no se han actualizado con el paso del tiempo, se ha encontrado instalado software no apropiado para el negocio de la empresa y no se han registrado las nuevas adquisiciones de aplicaciones y licencias. Esto conlleva que se realicen, en algunos casos, gastos totalmente innecesarios.  
        \item \textbf{Ausencia de medidas de seguridad f�sicas:} no existen medidas que controlen o limiten el acceso a las instalaciones de la empresa y a sus diversos departamentos. Esto es un grave riesgo para la seguridad de las instalaciones y los dispositivos que hay en la empresa.
        \item \textbf{Mala elecci�n y dise�o del centro de datos:} la ubicaci�n del servidor no cumple con las necesidades energ�ticas y de refrigeraci�n necesarias. Adem�s, ante un desastre ambiental o un incendio, el servidor estar�a desprotegido y podr�a resultar da�ado, llegando incluso a perder la informaci�n que contiene.
        \item \textbf{Ausencia de medidas de administraci�n y protecci�n de datos:} no se han definido procedimientos adecuados para el almacenamiento, copia y protecci�n de los datos que posee la empresa: el acceso a la informaci�n no est� controlado, las copias de respaldo no se tratan de forma correcta y, a la hora de eliminar datos, no se procede adecuadamente.
        \item \textbf{Ausencia de un plan de evaluaci�n y gesti�n de riesgos formalizado:} aunque existe un plan de gesti�n y evaluaci�n de riesgos, no es v�lido, ya que fue elaborado por algunos de los miembros del Consejo de Administraci�n y no existe ning�n informe en el cual se indique qui�nes de estos miembros estaban presentes ni las personas a las que consultaron a la hora de realizarlo. Adem�s, no se tratan mucho de los riesgos, lo que puede provocar grandes p�rdidas a la empresa ante una posible cat�trofe o contingencia.
        
\end{milista}

Atentamente, 

\begin{flushright}
	\textbf{\textit{ExtAudi}}: Empresa especializada en Auditor�as Externas.
\end{flushright}
