\begin {spacing} {1.4}
Estimado Sr. Director General:

Tras realizar la auditor�a a la sede de su empresa DIGSOL en Toledo,
 tal y como nos solicit�, le resumimos los problemas m�s
 graves que se han encontrado y que requerir�an atenci�n por parte del Comit� de Direcci�n: 

\begin{milista}
	\item \textbf{Ausencia de un plan estrat�gico de Tecnolog�as de la Informaci�n:} no existe una integraci�n total de las Tecnolog�as de la Informaci�n en la empresa, lo que conlleva que no se aprovechen correctamente las tecnolog�as, no se investigue en ellas para obtener mejores beneficios para la empresa, se realicen inversiones innecesarias y se prioricen incorrectamente los proyectos inform�ticos.
	\item \textbf{Ausencia de un plan de Evaluaci�n y Gesti�n de Riesgos formalizado:} aunque existe un plan de Gesti�n y Evaluaci�n de riesgos no es v�lido ya que no contempla muchos de los posibles riesgos existentes. Adem�s fue elaborado por algunos de los miembros del Consejo de Administraci�n y no existe ning�n informe en el cual se indique qui�nes de estos miembros estaban presentes en la elaboraci�n del plan ni las personas a las que consultaron a la hora de realizarlo. Esto puede provocar grandes p�rdidas a la empresa ante una posible cat�trofe o contingencia.
        \item \textbf{Gesti�n de la configuraci�n y control de cambios: } los documentos que definen la configuraci�n inicial de los equipos no se han actualizado con el paso del tiempo y se ha encontrado instalado software no apropiado para el negocio de la empresa. Esto hace que la empresa sea vulnerable a ataques inform�ticos. Adem�s, no se han registrado las nuevas adquisiciones de aplicaciones y licencias y se han vuelto a comprar en reiteradas ocasiones, siendo, en algunos casos, gasto totalmente innecesario.

\end{milista}
\end{spacing}