% -*- coding: utf-8 -*-

\subsection{Problemas relacionados con la adquisici�n de recursos de TI}

\begin{spacing}{1.4}
Estudiando el �rea de distribuci�n, almacenaje, venta y catalogado de la empresa DIGSOL, hemos observado que no existe un control en el seguimiento de unos procedimientos a la hora de adquirir nuevos productos, tanto software como hardware.
Se ha observado que no existe un procedimiento claro a la hora de adquirir productos nuevos para su posterior venta por internet. Una de las cosas m�s llamativas es que el encargado del almac�n es el que se encarga de realizar los pedidos de material en funci�n de las existencias que vayan quedando en el almac�n. Adem�s ninguno de esos pedidos es consultado previamente con el departamento directivo.
Otra de las cosas que deben cambiar o mejorar es el hecho de no tener informatizada una lista de proveedores en una base de datos, por lo que suponemos que los pedidos se realizan siempre a un mismo proveedor, siendo esto una clara deficiencia en el terreno de la competitividad con otras empresas del sector, ya que, no se pueden comparar precios de un mismo producto entre distintos proveedores o, incluso, no se puede comprobar si existen productos mejorados o m�s actualizados en el mercado.
Creemos que es conveniente elaborar un plan estrat�gico a seguir a la hora de realizar nuevos pedidos, comenzando por una jerarqu�a de revisiones y aprovaciones que han de seguir y cumplir esos pedidos; siguiendo por un control del presupuesto necesario para la realizaci�n del mismo, por lo que previamente se ha de informatizar una lista detallada de proveedores, productos y precios en una base de datos, buscando mejorar los beneficios de la empresa con la venta de productos por internet, aumentando los beneficios y reduciendo los costes.

\end{spacing}