% -*- coding: utf-8 -*-

\subsection{Problemas relacionados con Adquirir y Mantener Infraestructura Tecnol�gica}

\begin{spacing}{1.4}

% Relacionado con el punto AI3.1
Como ya se ha comentado, la empresa carece de un Departamento de Sistemas de Informaci�n que pueda, entre otras cosas, elaborar un plan para la adquisici�n y mantenimiento de recursos tecnol�gicos. Por ello, cuando cada departamento solicita nuevas tecnolog�as, la direcci�n lo aprueba, ya que no hay ning�n responsable que pueda evaluar si realmente la adquisici�n de hardware es necesaria. Por esta raz�n, la inversi�n en tecnolog�a que realiza la empresa es m�s elevada de lo necesario, lo que puede traducirse en p�rdidas econ�micas importantes a largo plazo. En el Anexo ?? (p�gina \pageref{anexo:facturas}) puede verse un ejemplo de gastos innecesarios en tecnolog�as.

Adem�s, tampoco se realiza un seguimiento ni una evaluaci�n para comprobar que los nuevos recursos obtenidos (principalmete hardware) est�n aportando beneficio a la empresa y est�n ayudando a cumplir su objetivo de negocio. 

\end{spacing}