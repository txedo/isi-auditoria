\subsection{Problemas relacionados con la Planificaci�n Estrat�gica de TI}

El objetivo a largo plazo de la empresa DIGSOL, definido en su plan estrat�gico, es mantenerse entre las empresas m�s competitivas del mercado.

Sin embargo, estudiando el plan estrat�gico cedido por el Director General de la empresa, as� como el organigrama, no se han establecido los procesos necesarios para cumplir dicho objetivo, ya que no cuenta con un departamento de I+D ni con un departamento de Tecnolog�as de la Informaci�n. Por este motivo, el objetivo a largo plazo de la empresa no puede cumplirse, ya que para poder seguir siendo competitiva, debe investigar nuevas tecnolog�as y actualizar las que ya tiene.

Otro problema, relacionado con la inexistencia del departamento de Tecnolog�as de la Informaci�n, es que cada departamento solicita nuevas tecnolog�as frecuentemente ya que los empleados no tienen los conocimientos suficientes para poder explotar al m�ximo las tecnolog�as actuales. Esto provoca que la inversi�n en tecnolog�a sea m�s elevada de lo necesario, ya que la Direcci�n aprueba las paeticiones de los distintos departamentos.

En cuanto a la alineaci�n de los objetivos del negocio, 


