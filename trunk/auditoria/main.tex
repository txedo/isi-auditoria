% Tipo de documento. En este caso es un art�culo, para folios A4, tama�o de la fuente 11pt y con p�gina separada para el t�tulo
\documentclass[a4paper,12pt,titlepage]{article}

% Carga de paquetes necesarios. OrdenesArticle es un paquete personalizado
\usepackage[spanish]{babel} 
\usepackage[T1]{fontenc}
\usepackage[ansinew]{inputenx} 
\usepackage[spanish,cap,cont,title,fancy]{OrdenesArticle}
\usepackage{setspace}
\usepackage{array}
\usepackage{graphicx}
\usepackage{hyperref}
\usepackage{pifont}
\usepackage{listings}
\usepackage[usenames,dvipsnames]{color}
\usepackage{colortbl}
\usepackage{makeidx}
\hypersetup{bookmarksopen,bookmarksopenlevel=3,linktocpage,colorlinks,urlcolor=blue,citecolor=blue,
						linkcolor=blue,filecolor=blue,pdfnewwindow,
						pdftitle={Auditor�a a DIGSOL}, 
						pdfauthor={Juan Andrada, Jose Domingo L�pez, Antonio Mart�n Menor, Francisco Jos� Oteo}}


% Macro para definir una lista personalizada 
\newenvironment{milista}%
{\begin{list}{\textbullet}%
{\settowidth{\labelwidth}{\textbullet} \setlength{\leftmargin}{\dimexpr\labelsep+\labelwidth+5pt}
\setlength{\itemsep}{\dimexpr 0.5ex plus 0.25ex minus 0.25ex}
\setlength{\parsep}{\itemsep}
\setlength{\partopsep}{\itemsep}
\addtolength{\topsep}{-7.5pt}
}}%
{\end{list}}

\begin{document}

% En las p�ginas de portada e �ndices, no hay encabezado ni pie de p�gina
\pagestyle{fancy} 

% Se incluye la portada
\begin{titlepage}
	\begin{center}
  	{\LARGE UNIVERSIDAD DE CASTILLA-LA MANCHA} \\
  	\bigskip
  	{\Large ESCUELA SUPERIOR DE INFORM�TICA} \\
  	\vspace{20mm}
  	\includegraphics[scale=0.45, keepaspectratio]{esi_bw.png} \\
  	\vspace{20mm}
  	{\Huge \textbf{Auditor�a y Seguridad de la Informaci�n}} \\
  	\vspace{10mm}
  	{\LARGE \textsc{\textbf{Auditor�a a DIGSOL}}} \\
  	\vspace{30mm}
  	\today
  	\vspace{30mm}
  	\begin{flushleft}
  		{\large Juan Andrada Romero}\\
  		\vspace{1mm}
  		{\large Jose Domingo L�pez L�pez}\\
  		\vspace{1mm}
  		{\large Antonio Mart�n Menor de Santos}\\
  		\vspace{1mm}
  		{\large Francisco Jos� Oteo Fern�ndez}\\  		
  	\end{flushleft}
	\end{center}
\end{titlepage}

% Se ajusta la separaci�n entre p�rrafos
\parskip=10pt

% Aqui se incluyen los archivos .tex que forman el documento
\section{Carta al Director}
\input{CartaDirector.tex}
\clearpage

\section{Auditor�a}
\subsection{Ausencia de un plan estrat�gico de Tecnolog�as de la Informaci�n}

\begin{spacing}{1.4}

Consultando al Comit� de Direcci�n de la empresa sobre la existencia de un plan estrat�gico de TI (Tecnolog�as de la Informaci�n), se ha observado que no existe dicho plan. Esto significa que, a la hora de intentar cumplir los objetivos de negocio que la empresa tiene fijados, las tecnolog�as no son algo importante para conseguirlos, si no una simple ayuda \textbf{(ver Anexo 1)}.\\
\indent Es decir, la empresa no est� alineada con la tecnolog�a, lo que puede suponer, a largo o medio plazo, que la empresa comience a perder presencia en el mercado, ya que para seguir siendo competitiva, es necesario investigar nuevas tecnolog�as, actualizar los recursos tecnol�gicos con los que ya cuenta la empresa y aprovechar las oportunidades que ofrecen las tecnolog�as de la informaci�n. En el \textbf{Anexo 2} se muestra la inversi�n que las empresas de la competencia realizan en I+D de nuevas tecnolog�as y la situaci�n actual en el mercado. Como se puede observar, si la empresa no alinea las tecnolog�as con su objetivo de negocio, en poco tiempo perder� la posici�n en el mercado con respecto a su competidora m�s inmediata.

Derivado de la inexistencia de un plan estrat�gico de TI que alinee las tecnolog�as con el objetivo de negocio, aparece otro problema, que es la falta de un Departamento de Sistemas de Informaci�n en la organizaci�n de la empresa, tal y como se observa en el \textbf{Anexo 3}. Como consecuencia, los desarrolladores inform�ticos est�n asociados a los diferentes departamentos de la empresa, desarrollando los proyectos que sean necesarios en cada momento (entre otras cosas), pero sin contar con un responsable experto en TI.

Al carecer de un Departamento de Sistemas de Informaci�n, no hay ning�n responsable que se encargue de elaborar un plan para la adquisici�n y mantenimiento de recursos tecnol�gicos \textbf{(ver Anexo 1)}. Por ello, como la empresa carece de dicho plan, cuando cada departamento solicita nuevos recursos tecnol�gicos, el departamento de contabilidad aprueba la inversi�n, bajo la orden del Comit� de Direcci�n. Por esta raz�n, la inversi�n en tecnolog�a que realiza la empresa es m�s elevada de lo necesario, lo que puede traducirse en p�rdidas econ�micas importantes a largo plazo. Adem�s, los recursos se adquieren de diversos proveedores, lo que pone en riesgo la compatibilidad, tanto de hardware como de software. Por otra parte, como el Comit� de Direcci�n aprueba la inversi�n siempre que se solicitan nuevos recursos, existe el riesgo de que cualquier desarrollador cometa fraudes, pudiendo establecer como proveedor a uno con el que haya pactado alguna comisi�n o comprando recursos que �l desee y que sean totalmente innecesarios. En el \textbf{Anexo 4} puede verse un ejemplo de estas situaciones.

Otro problema es que no existe un procedimiento por el que se sigan unas pautas a la hora de realizar pagos o devoluciones de los recursos adquiridos, as� como tampoco los contratos firmados con los proveedores han sido revisados por un asesor legal, ni se indican cl�usulas u obligaciones que deben cumplirse por parte de los proveedores \textbf{(ver Anexo 1)}. Al no establecerse esas cl�usulas contractuales, si un proveedor no satisface un pedido en el per�odo acordado, puede repercutir en graves p�rdidas econ�micas, lo que pone en riesgo la disponibilidad de alg�n recurso y, por tanto, el correcto funcionamiento de la empresa, con las correspondientes p�rdidas econ�micas. 

Al igual que no existe un plan para la adquisici�n de infraestructura tecnol�gica, tampoco existe un plan para su mantenimiento, como muestran los Anexos 1 y 5. Simplemente, los encargados de inform�tica de cada departamento realizan las labores de mantenimiento necesarias cuando ocurre alg�n problema, al igual que se encargan de aplicar las actualizaciones necesarias al software instalado. Esto conlleva un descontrol, ya que no se relizan documentos acerca del mantenimiento realizado ni se sigue un control de cambios en el software que utiliza la empresa.

Para terminar, hay que se�alar otro problema. Al no existir el plan estrat�gico que alinee la empresa con las TI, tampoco existen portafolios de proyectos (esto es, lista de proyectos a desarrollar en la empresa) que asigne correctamente las prioridades a los diferentes proyectos que maneja la empresa. As�, los desarrolladores inform�ticos se limitan a desarrollar las aplicaciones que dichos departamentos necesiten para llevar a cabo su tarea, en el orden en el que los empleados las van solicitando \textbf{(ver Anexo 5)}.\\
\indent Esto tiene el riesgo de que no se atiendan a tiempo los proyectos m�s prioritarios para mantener el objetivo de negocio de la empresa y que, por tanto, se produzcan p�rdidas econ�micas, adem�s de perder competitividad en el mercado.

\end{spacing}

%% -*- coding: utf-8 -*-

\subsection{Problemas relacionados con la Planificaci�n Estrat�gica de TI}

\begin{spacing}{1.4}

El objetivo a largo plazo de la empresa DIGSOL, definido en su plan estrat�gico, es mantenerse entre las empresas m�s competitivas del mercado.

Sin embargo, estudiando el plan estrat�gico cedido por el Director General de la empresa, as� como el organigrama, no se han establecido los procesos necesarios para cumplir dicho objetivo, ya que la empresa no cuenta con un departamento de I+D ni con un departamento de Tecnolog�as de la Informaci�n. Por este motivo, a medio o largo plazo, el objetivo de negocio de la empresa corre un riesgo y puede suponer p�rdidas econ�micas importantes, ya que, para que la empresa pueda seguir siendo competitiva, debe investigar nuevas tecnolog�as y actualizar las que ya tiene. De lo contrario, el actual sistema gestor de base de datos, as� como el resto del sistema, se quedar�a obsoleto y el resto de empresas ganar�an mercado  al satisfacer las necesidades de los clientes de una manera m�s rapida y eficiente.

Otro problema, relacionado con la inexistencia del departamento de Tecnolog�as de la Informaci�n, es que cada departamento solicita nuevas tecnolog�as frecuentemente, ya que no existe un director experto en Tecnolog�as de Informaci�n que pueda controlar a los distintos departamentos y sacar el m�ximo provecho a las tecnolog�as existentes. Esto provoca que la inversi�n en tecnolog�a sea m�s elevada de lo necesario, ya que la Direcci�n aprueba las peticiones de los distintos departamentos.

Como ya se ha dicho, siguiendo el organigrama de la empresa, �sta carece de un departamento de Tecnolog�as de la Informaci�n, por lo que los desarrolladores inform�ticos est�n asociados a los distintos departamentos de la empresa y su trabajo consiste en desarrollar las aplicaciones que dichos departamentos necesiten para llevar a cabo su tarea. Esto supone un problema en cuanto a las prioridades del negocio, ya que los proyectos se van desarrollando en el orden en que son propuestos por los distintos departamentos (tras la aprobaci�n de la Direcci�n). Esto puede provocar que se desarrollen en primer lugar proyectos de menos importancia en la empresa y que se demoren los proyectos m�s prioritarios, lo que puede suponer, a medio o largo plazo, p�rdidas econ�micas para la empresa.

\end{spacing}




\clearpage
\subsection {Evaluaci�n de riesgos, plan de contingencia y plan de continuidad}

\begin{spacing}{1.4}

La evaluaci�n y gesti�n de riesgos es un tema que no est� abordado de
una forma eficiente en la organizaci�n.

Aunque la empresa dispone de un plan de riesgos, �ste ha sido elaborado por algunos de
los miembros del Consejo de Administraci�n y no existe ning�n informe
en el cual se indique qui�nes de estos miembros estaban presentes en la
elaboraci�n del plan ni las personas a las que consultaron a la hora
de realizarlo. 

Adem�s, tampoco existe ning�n registro de la t�cnica de
identificaci�n de riesgos que emplearon para seleccionarlos, as� como el 
c�lculo de la probabilidad de ocurrencia y el impacto, lo cual puede 
significar que la elecci�n de �stos par�metros fuese arbitraria o manipulada. 
En el Anexo 22 se muestra el acta de los asistentes a dicha reuni�n.

La empresa por tanto, no tiene definido un plan de continuidad de TI por lo que en caso de interrupci�n de los sistemas el impacto ser� mayor y en consecuencia se ver�n afectados
los procesos clave de negocio.

De lo anterior se extrae que la empresa tendr�a dificultades para volver a su actividad normal en caso de un desastre o contingencia lo que podr�a provocar perdidas muy grandes para la 
empresa.

Las copias de seguridad que posee la empresa no son suficientemente frecuentes. Adem�s carece de copias de seguridad fuera de las instalaciones por lo que en caso de desastre en las instalaciones
existe el riesgo de perder toda la informaci�n relevante de la empresa. En el Anexo 23 se muestra c�mo se almacenan algunas de las copias de seguridad.

\end{spacing}
%% -*- coding: utf-8 -*-

\subsection{Problemas relacionados con la Arquitectura de la Informaci�n}

\begin{spacing}{1.4}

%En cuanto a diccionario de datos empresarial y reglas de sintaxis

La empresa DIGSOL carece de un diccionario de datos empresarial que defina las reglas
de sintaxis para los datos de la organizaci�n lo que produce:
-Peor entendimiento entre los usuarios del sistema de TI y del Negocio.
-Posibilidad de creaci�n de elementos de datos incompatibles.

%En cuanto al esquema de clasificacion de datos y la administraci�n de la integridad
Existe un esquema de clasificaci�n de datos definido en el que cada tipo de usuario tiene
acceso a una informaci�n concreta pero al no encriptar la informaci�n que viaja en texto plano
por la red local es f�cilmente interceptable por los dem�s usuarios de la empresa incluso por usuarios
externos ya que adem�s la protecci�n de la wifi se realiza mediante clave WEP que es un tipo de clave 
f�cil de obtener mediante aplicaciones como AirCrack lo que compromete la integridad del sistema.

Los USB y los dispositivos �pticos de los ordenadores de la empresa no 
est�n bloqueados, lo que permitir�a que un usuario malintencionado tomar documentos
 y datos de car�cter confidencial para la empresa, introducir virus o troyanos en el sistema.
%En cuanto a Administraci�n de integridad

No existen procedimientos peri�dicos para cambio de contrase�as de los
usuarios en el sistema. Estas claves son facilitadas en la incorporaci�n del
usuario a la empresa y no son eliminadas tras su despido por el departamento de
SI cuando recibe la confirmaci�n pertinente.

El diagrama de la base de datos es correcto pero su implementaci�n no es coherente con el diagrama dise�ado.
Una vez hemos estudiado la base de datos hemos comprobado que las claves ajenas est�n ausente en la creaci�n 
de las tablas, as� como campos que no deben estar vacios est�n definidos con la posibilidad de serlo.

El acceso a la base de datos es posible si se conoce la cadena de conexi�n a la misma ya que carece de clave de
usuario y de acceso ya que se ha utilizado SQL Server con modo de autenticaci�n de Windows. Es posible obtener 
la ruta de acceso a la base de datos utilizando aplicaciones sniffer que se utilizan para escuchar la 
informaci�n que viaja por nuestra red local de forma que podemos obtener m�s informaci�n de la que el usuario
debiese obtener.

Carece de una base de datos centralizada que contenga la informaci�n completa de la empresa, hay una base de datos
que almacena una parte (productos,empleados,usuarios registrados...) y otra parte que se almacena en ficheros de texto
como facturas... Puesto que las facturas est�n relacionadas con productos, usuarios registrados y empleados entre otros
estas deber�an estar integradas en la base de datos centralizada.



\end{spacing}


\clearpage
%\begin{spacing}{1.4}

La evaluación y gestión de riesgos es un tema que no está abordado de
una forma eficiente en la organización ya que, aunque la empresa
dispone de un plan de riesgos, éste ha sido elaborado por algunos de
los miembros del Consejo de Administración y no existe ningún informe
en el cual se indique quiénes de estos miembros estaban presentes en la
elaboración del plan ni las personas a las que consultaron a la hora
de realizarlo. Además, tampoco existe ningún registro de la técnica de
identificación de riesgos que emplearon (p.e. Delphi) para
seleccionarlos, así como el cálculo de la probabilidad de ocurrencia y
el impacto, lo cual puede significar que la elección de éstos parámetros
fuese arbitraria o manipulada. En el Anexo ?? se muestra el acta de
los asistentes a dicha reunión. 

Un aspecto a destacar en los riesgos seleccionados y planificados es
que se ciñen al ámbito software dejándo de lado el ámbito hardware, no
teniendo en  cuenta, por ejemplo, factores externos sobre los cuales
la organización puede no tener ningún control debido a su naturaleza
(terremotos, colisiones de aviones, etc...) o incluso el robo de material o
incendios. En el Anexo ?? se puede observar una lista de los riesgos
que deberían ser revisados y/o considerados.

Además, el CIO y el Consejo de Administración han ido elaborando una
lista de los nuevos riesgos que han ido azotando a la organización
pero no los han documentado en términos del impacto que éstos
produjeron ni las estrategias de mitigación o planes de contigencia
que se llevaron a cabo para combatirlos con el objetivo de reutilizarlos
o de mejorarlos. Debido a ésto, algunos riesgos se han producido
en varias ocasiones y no se ha ido reduciendo el impacto. Como prueba
de ello, en el Anexo ?? se puede apreciar que la empresa ha sufrido
dos apagones de luz generales en seis meses, las pérdidas económicas
que supusieron ámbos, el coste de haber desarrollado un plan de
contigencia, y las pérdidas que se habrían producido si dicho plan se
hubiese llevado a cabo.

\end{spacing}
\subsection {Gesti�n de la configuraci�n y control de cambios}

\begin{spacing}{1.4}

La empresa dispone de una herramienta de soporte y un repositorio central para almacenar informaci�n relevante sobre los elementos de configuraci�n hardware y software. El objetivo es garantizar la integridad de las configuraciones, pero este repositorio no se encuentra actualizado con las nuevas adquisiciones y necesidades de la empresa, ni con la evoluci�n de las TI.

En el Anexo ?? se observa un fragmento del documento que especifica la configuraci�n inicial de los equipos que utilizar�n los empleados. Como se puede apreciar, el software que inicialmente se instalar� en estos equipos est� obsoleto y no existe una norma est�ndarice la instalaci�n de parches y actualizaciones de seguridad, as� como la configuraci�n de servicios y par�metros del sistema. Esto hace que los sistemas no sean seguros y puedan ser vulnerables a ataques inform�ticos (ver Anexo ??).

Otra de las consecuencias que acarrea no tener actualizado el repositorio afecta al uso y adquisici�n de licencias. En el Anexo ?? se puede apreciar que en el �ltimo a�o se han adquirido licencias duplicadas para el antivirus y que a�n as� existen equipos con licencias caducadas. Este problema, s�lo en el caso del antivirus, ha supuesto un total de XXXXXX euros.

Por otro lado, los cambios realizados no son registrados en ning�n documento ni previamente autorizados por el Director de Inform�tica, y no se mantiene una l�nea base de los elementos de la configuraci�n para todos los sistemas y servicios como punto de comprobaci�n al que volver tras el cambio. Adem�s, no se garantizan los resultados de los cambios antes de realizarse ya que �stos no est�n sujetos a ning�n tipo de norma y cada usuario los hace libremente. Como prueba de ello, en el Anexo ?? se observan capturas de pantalla de software personal y/o no licenciado que ha sido instalado en los equipos, acci�n que podr�a haber sido corregida si los equipos se hubiesen revisado peri�dicamente por el Gerente de Configuraci�n.

\end{spacing}


\clearpage
%% -*- coding: utf-8 -*-

\subsection{Problemas relacionados con Adquirir y Mantener Infraestructura Tecnol�gica}

\begin{spacing}{1.4}

% Relacionado con el punto AI3.1
Como ya se ha comentado, la empresa carece de un Departamento de Sistemas de Informaci�n que pueda, entre otras cosas, elaborar un plan para la adquisici�n y mantenimiento de recursos tecnol�gicos. Por ello, como la empresa carece de dicho plan, cuando cada departamento solicita nuevas tecnolog�as, la direcci�n lo aprueba. Por esta raz�n, la inversi�n en tecnolog�a que realiza la empresa es m�s elevada de lo necesario, lo que puede traducirse en p�rdidas econ�micas importantes a largo plazo. En el Anexo ?? (p�gina \pageref{anexo:facturas}) puede verse un ejemplo de gastos innecesarios en tecnolog�as. 

% Relacionado con el punto AI3.2
En lo que se refiere a la protecci�n de los recursos de infraestructura, realizando una visita por las instalaciones de la sede, se ha comprobado que la seguridad es m�nima. En primer lugar, los equipos de cada uno de los departamentos carecen de medidas antirrobo y se pueden manipular de una manera sencilla para sustraer partes importantes, como un disco duro (ver anexo ??) % Se puede hacer una foto de ordenadores abiertos y sin candados
Adem�s, el cableado de red de los diferentes equipos es visible, por lo que su desconexi�n y manipulaci�n es simple (ver Anexo ??) % Foto de cables

Por otra parte, los servidores residen en una sala a la que pueden acceder los desarrolladores inform�ticos de los distintos departamentos, lo que pone en riesgo la seguridad, integridad y disponibilidad de los servidores, ya que cualquiera de ellos puede manipularlos sin el consentimiento ni el conocimiento del resto. Adem�s, los servidores no cuentan con SAIs (Sistemas de Alimentaci�n Ininterrumpida), por lo que, frente a un fallo el�ctrico, la empresa quedar�a paralizada. % Foto de un armario de servidores sin proteccion ni SAI

Todo esto compromete la disponibilidad y seguridad de la infraestructura hardware, ya que la manipulaci�n de algunos recursos puede hacerse de una manera simple y sin controles internos que lo eviten.

Atendiendo a otros temas, tampoco se realiza un seguimiento y evaluaci�n de los recursos adquiridos, por lo que realmente la empresa no conoce si la inversi�n que realiza en tecnolog�a est� otorgando beneficio y s� realmente las nuevas tecnolog�as ayudan a alcanzar el objetivo de negocio de la empresa. 

% Relacionado con el punto AI3.3
Al igual que no existe un plan para la adquisici�n de infraestructura tecnol�gica, tampoco existe un plan para su mantenimiento. Simplemente, los encargados de inform�tica de cada departamento realizan las labores de mantenimiento necesarias cuando ocurre alg�n problema, al igual que se encargan de aplicar las actualizaciones necesarias al software instalado. Esto conlleva un descontrol, ya que no se relizan documentos acerca del mantenimiento realizado ni se sigue un control de cambios en el software que utiliza la empresa.


\end{spacing}
% -*- coding: utf-8 -*-

\subsection{Deficiencia en la seguridad de la informaci�n y en las instalaciones}

\begin{spacing}{1.4}

Continuando con la auditor�a, con respecto a las instalaciones de la empresa, se observa que:
	- Las instalaciones son poco o nada seguras.
	- Existen deficiencias en la ubicaci�n del servidor.
	
	Observando el entorno de trabajo y las instalaciones que posee la empresa se ve que carece de cualquier tipo de medida sobre la restricci�n o no del acceso a las instalaciones por parte del personal como por parte de personal ajeno a la empresa.
	Todas las salas en las que se ubican los departamentos carecen de cerraduras en las puertas o de sistemas de acceso controlado, tampoco se controla el acceso a la empresa de ninguna manera. Cualquiera puede entrar a las instalaciones de la empresa y acceder a cada departamento pudiendo llevarse f�sicamente cualquier dispositivo que contenga informaci�n, lo que provoca un grave riesgo en la seguridad.
	Cuando se pregunta si exist�a y se pod�a revisar el plan de seguridad con las medidas relativas al acceso a las instalaciones se obtuvo una respuesta negativa.
	
	La ubicaci�n del centro de datos y del servidor que los contiene no ha sido bien planeada y estudiada ya que no se corresponde con el dise�o �ptimo que debe tener el espacio f�sico destinado a servidores, por lo que existe riesgo de que el servidor se caliente en exceso y se estropee dejando sin servicio de venta online a la empresa, lo que se traducir�a en cuantiosas p�rdidas econ�micas, tanto en ventas como en deterioro de la imagen de la empresa en internet.	
En caso de incendio, inundaci�n o alg�n otro desastre, el servidor que contiene los datos podr�a estropearse, ya que, como se ha observado, el local en el que se ubica no es el apropiado ni est� dotado de medidas de prevenci�n y protecci�n ante tales eventualidades.

	Finalmente hay que destacar que no existe ning�n sistema de alimentaci�n ininterrumpida ni tampoco una fuente de corriente alternativa, por lo que cualquier corte de luz llevar�a forzosamente a una suspensi�n del servicio de venta por internet as� como del acceso a datos de otra �ndole, por no hablar de posibles da�os en el servidor y p�rdida de datos.
	
Con respecto a medidas de seguridad f�sicas para limitar o impedir el acceso a dispositivos y a la informaci�n que estos contienen se puede observar que el servidor no se encuentra dentro de un armario de datos bajo llave, sino que se encuentra en una estanter�a junto con otros ordenadores.
	Tanto las ranuras de expansi�n USB, como la grabadora de DVD que contiene el servidor est�n visibles y accesibles, pero adem�s, la parte posterior del servidor est� tambi�n accesible y pueden observarse las tarjetas de red. Si alg�n empleado malintencionado quisiera llevarse datos podr�a hacerlo pinchando una memoria USB, grabando un DVD o conectando otro PC a una tarjeta de red del servidor, interrumpiendo adem�s el servicio web o de acceso a los datos. \textbf{(Ver anexo servidor)}.
	
	La ausencia de un cortafuegos f�sico que limite el acceso al mismo desde fuera es otra deficiencia de la seguridad importante, ya que puede dar lugar a que datos de car�cter personal sobre los empleados de la empresa, cuentas, claves, etc. pueden ser accedidos desde fuera de la empresa v�a web.
	
Cuando se pregunt� al personal responsable de inform�tica sobre qu� procedimientos se segu�an para el correcto procesamiento y almacenamiento de los datos de la empresa se nos contest� que s� se hab�a definido un responsable para administrar esos datos y su almacenamiento, pero que no se le hab�a informado sobre la existencia de ning�n procedimiento, por lo que se hab�a optado por almacenar todos los datos en un mismo servidor, incluso en un mismo directorio dentro de �ste.
Indagando un poco m�s se descubri� que s� se realizaban copias de seguridad de los datos, tanto los referidos a la informaci�n sensible a la empresa, como la informaci�n de los productos que venden; lamentablemente el soporte en el que se almacena la copia de respaldo de esos datos es un disco duro externo que se guarda en un caj�n, sin vigilancia ni medidas de seguridad, del mismo edificio en el que se encuentra el servidor. \textbf{(Ver anexo caj�n con llaves)}.

	A parte del encargado de administrar los datos, nadie accede, en principio, a los datos, pero no se toman las medidas de seguridad apropiadas para asegurarse de que no accedan a los datos personas no autorizadas. De hecho se ha preguntado por los ficheros de \textit{log}, pero examinando los logs de accesos al sistema se ha detectado que existen accesos al servidor por empleados que usan la cuenta -empleado- y no est�n relacionados con datos inform�ticos, incluso puede que alguno de esos accesos se hayan hecho por empleados que fueron despedidos, pero no se puede probar ya que todos los empleados usan la misma cuenta. Se advierte de que la informaci�n que dichos usuarios puedan extraer de la empresa podr�a ser vendida a la competencia. Adem�s, dichos logs corresponden �nicamente a los �ltimos 3 meses de actividad, siendo de 2 a�os el periodo exigido por la ley. (Ver anexo logs).

	Respecto a ficheros de datos, cuyo soporte no es el electr�nico, se encuentra una estanter�a llena de papeles con informaci�n relevante para la empresa. Esta estanter�a no est� protegida bajo llave u otros sistemas de seguridad, y adem�s mezcla datos y cajas u otras cosas sin un orden aparente. Aqu� existe tambi�n un serio problema de seguridad, ya que la informaci�n contenida en esos papeles puede ser f�cilmente accedida. (Ver anexo estanter�a).
	
	Estudiando los comportamientos de seguridad de los empleados se observa que:
		- El estado de sus ordenadores no es el adecuado. Tienen los ordenadores abiertos f�sicamente, dejando al descubierto la electr�nica y los dispositivos de almacenamiento de datos (como el disco duro), siendo relativamente f�cil da�ar un equipo o robar informaci�n del mismo. \textbf{(Ver anexo PC abierto)}.
		- Por otro lado, cuando un empleado abandona su puesto de trabajo, �ste no bloquea el acceso al mismo, por lo que cualquier otro empleado podr�a usar este ordenador con fines malintencionados.

	A la hora de eliminar o destruir datos existen dos v�as para hacerlo: si los datos se encuentran en soporte electr�nico, es el administrador de datos el que los borra; y, si los datos est�n en papel, para eliminarlos se rompe el papel y se tira a la papelera. En cuanto al procedimiento a seguir cuando se estropea un soporte de almacenamiento, no est� especificado, por lo que simplemente, o bien se tira a la basura o bien se lleva a reciclar, pero no se destruye la informaci�n que contiene, por lo que alguien podr�a recuperar esos datos de alguna manera usando un lector magn�tico, etc. \textbf{(Ver anexos papelera)}.
	
	Tampoco se han encontrado existencias de un libro de incidencias, tal y como dicta el art�culo 10 del Real Decreto 994/99, en que hagan constar los siguientes puntos:
		a) el tipo de incidencia
		b) El momento en que se produce
		c) La persona que realiza la notificaci�n
		d) A qui�n se le comunica
		e) Los efectos que se deriven de la incidencia.

\end{spacing}
\clearpage
%% -*- coding: utf-8 -*-

\subsection{Problemas relacionados con la adquisici�n de recursos de TI}

\begin{spacing}{1.4}
Estudiando el �rea de distribuci�n, almacenaje, venta y catalogado de la empresa DIGSOL, hemos observado que no existe un control en el seguimiento de unos procedimientos a la hora de adquirir nuevos productos, tanto software como hardware.
Se ha observado que no existe un procedimiento claro a la hora de adquirir productos nuevos para su posterior venta por internet. Una de las cosas m�s llamativas es que el encargado del almac�n es el que se encarga de realizar los pedidos de material en funci�n de las existencias que vayan quedando en el almac�n. Adem�s ninguno de esos pedidos es consultado previamente con el departamento directivo.
Otra de las cosas que deben cambiar o mejorar es el hecho de no tener informatizada una lista de proveedores en una base de datos, por lo que suponemos que los pedidos se realizan siempre a un mismo proveedor, siendo esto una clara deficiencia en el terreno de la competitividad con otras empresas del sector, ya que, no se pueden comparar precios de un mismo producto entre distintos proveedores o, incluso, no se puede comprobar si existen productos mejorados o m�s actualizados en el mercado.
Creemos que es conveniente elaborar un plan estrat�gico a seguir a la hora de realizar nuevos pedidos, comenzando por una jerarqu�a de revisiones y aprovaciones que han de seguir y cumplir esos pedidos; siguiendo por un control del presupuesto necesario para la realizaci�n del mismo, por lo que previamente se ha de informatizar una lista detallada de proveedores, productos y precios en una base de datos, buscando mejorar los beneficios de la empresa con la venta de productos por internet, aumentando los beneficios y reduciendo los costes.

\end{spacing}
\subsection{Riesgos para la integridad, disponibilidad y confidencialidad de los datos}

\begin{spacing}{1.4}

Como ya se ha comentado, la empresa carece de un Departamento de Sistemas de Informaci�n que pueda, entre otras cosas, elaborar un plan para la adquisici�n y mantenimiento de recursos tecnol�gicos. Es por ello, por lo que uno de los primeros riesgos que se observa es la ausencia de un segundo servidor alternativo.
Esto indica que existir�a un problema serio de disponibilidad de los datos ante una posible aver�a en el servidor principal, o ante un desastre, provocando una interrupci�n en el servicio de venta online y por lo tanto la falta de continuidad del negocio.

Otro punto d�bil que se ha encontrado es la existencia de una �nica cuenta de usuario dentro del servidor que sirve de puerta de acceso a este y a los datos que contiene, tanto para los empleados de la empresa relacionados con la parte inform�tica de cada departamento, como para el administrador del servidor y los usuarios de la web del servicio de venta de art�culos por internet. Esta cuenta se llama igual que el nombre de la empresa y el servidor, por lo que ser�a f�cil adivinar desde fuera cu�l podr�a ser un usuario del sistema y lanzar un ataque contra �l. \textbf{Ver anexo (cuentas de usuario).}

Adem�s, el resto de empleados que, en teor�a, no deber�an acceder al servidor, usan todos la misma cuenta de usuario, con la misma clave, y esa cuenta de usuario existe y est� operativa tambi�n dentro del servidor. Esta cuenta -empleados- pertenece al mismo grupo de usuarios que la cuenta -digisol-, por lo que se observa otro problema de seguridad, ya que cualquier empleado puede acceder al servidor; y, en caso de que un empleado acceda al servidor, no se sabr�a cu�l de ellos ha sido. 
Por otro lado, si un empleado con alg�n conocimiento de administraci�n quisiese modificar datos de los archivos no tendr�a grandes problemas debido a que varios archivos tienen permisos de lectura y escritura para el grupo de usuarios al que pertenecen el usuario -empleado- y el usuario -digsol-. \textbf{Ver anexo (cuentas de usuario y permisos archivos)}.

Otro problema es el hecho de que se use el servidor tanto para almacenar los datos referidos a los productos que se venden en la web, como datos sensibles sobre la empresa, referidos a la informaci�n de car�cter personal sobre empleados, clientes, n�minas, etc.
Adem�s ciertos archivos con informaci�n de car�cter personal y/o confidencial tienen unos permisos inadecuados, lo que evidencia un gran riesgo ya que cualquiera (ya sea o no usuario del sistema) puede tener acceso a esta informaci�n y leerla, modificarla o borrarla. Incluso existen ficheros en texto plano y sin cifrar que contienen las claves de acceso al sistema. Esto repercute tanto en la confidencialidad de los datos como en la integridad y disponibilidad de los mismos. Ver anexo (permisos en archivos).

Al consultar al administrador del servidor sobre la administraci�n y gesti�n de cuentas de usuario y sobre si existe una pol�tica definida a tal efecto, este nos comunica que no existe una pol�tica que obligue a los usuarios del sistema a cambiar su contrase�a personal en un periodo definido de tiempo, no pudiendo ser �sta la misma que una contrase�a anterior. Esto hace que si un usuario no autorizado consigue acceso mediante alguna cuenta, podr� hacerlo de por vida ya que la contrase�a no ser� modificada en un futuro pr�ximo.

Al igual que no existe un plan para la administraci�n de los datos, tampoco existe un plan para su mantenimiento. Simplemente, el encargado del servidor se encarga de dar de alta a los usuarios del sistema y asignarles claves; pero a su vez tambi�n se encarga de actualizar los datos de la web de venta de art�culos. Igualmente los encargados de inform�tica de cada departamento tienen acceso al servidor usando una �nica cuenta (como ya se ha comentado) y realizan las labores de mantenimiento necesarias con los datos y los archivos que los contienen, al igual que se encargan de aplicar las actualizaciones necesarias al software instalado. 
Esto conlleva un serio riesgo de integridad y seguridad de los datos, ya que cualquier trabajador podr�a alterar archivos de datos sin ning�n tipo de control sobre �l.

No se han creado planes para poder monitorear el desempe�o de las tecnolog�as de informaci�n ni para medir su capacidad. Puntualmente, los desarrolladores inform�ticos revisan los recursos para comprobar si pueden soportar la carga de trabajo de los distintos departamentos. Sin embargo, dichas revisiones no se documentan y las acciones tomadas se limitan a solicitar nuevos recursos tecnol�gicos si se observa que los actuales no soportan correctamente la carga de negocio.

Por tanto, no se monitorea la capacidad actual de los recursos, as� como tampoco se tiene en cuenta el posible crecimiento futuro de la empresa, lo que supondr� un aumento en la carga de trabajo, sobre todo para los servidores, existiendo un riesgo que pone en peligro la disponibilidad de los recursos.
Respecto a los diccionarios de datos que utiliza la empresa se nos comunica tras una serie de entrevistas que se carece de un diccionario de datos empresarial que defina las reglas de sintaxis para los datos de la organizaci�n lo que produce un peor entendimiento entre los usuarios del sistema y del negocio y la posibilidad de creaci�n de elementos de datos incompatibles, por lo que existe un riesgo ante la integridad de los datos.

Otro punto d�bil que afecta a la integridad de los datos es el hecho de que las comunicaciones por la red no est�n cifradas por lo que, al no encriptar la informaci�n (que viaja en texto plano por la red local), es f�cilmente interceptable por los dem�s usuarios de la empresa o incluso por usuarios externos, ya que adem�s la protecci�n de la wifi se realiza mediante clave WEP que es un tipo de clave f�cil de obtener mediante aplicaciones como AirCrack lo que compromete la integridad del sistema. \textbf{Ver anexo (crackear wifi)}.

Otro punto que afecta a la integridad de los datos es que, aunque el diagrama de la base de datos es correcto, su implementaci�n no es coherente con el diagrama dise�ado.
Una vez hemos estudiado la base de datos hemos comprobado que las claves ajenas est�n ausentes en la creaci�n de las tablas, as� como campos, que no deber�an estar vacios, est�n definidos con la posibilidad de serlo.

El acceso a la base de datos es posible, si se conoce la cadena de conexi�n a la misma, ya que carece de clave de usuario y de acceso al haberse utilizado SQL Server con modo de autenticaci�n de Windows. Es posible obtener la ruta de acceso a la base de datos utilizando aplicaciones de tipo sniffer que se utilizan para escuchar la informaci�n que viaja por nuestra red local.

Otro riesgo evidente para la integridad de la informaci�n es el hecho de carecer de una base de datos centralizada que contenga la informaci�n completa de la empresa. Hay una base de datos que almacena una parte (productos, datos de clientes, etc.) y otra parte que se almacena en ficheros de texto (facturas, etc.) o en ficheros de bases de datos Access o Excel. Puesto que las facturas est�n relacionadas con productos, usuarios registrados y empleados entre otros, estas deber�an estar integradas en la base de datos centralizada. \textbf{Ver anexo (permisos archivos)}.

\end{spacing}

\clearpage
%% -*- coding: utf-8 -*-

\subsection{Problemas relacionados con los cambios}

\begin{spacing}{1.4}

La empresa carece de procediminetos de administraci�n de cambios formales para manejar de manera est�ndar todas las solicitudes que afectan a:
\begin{enumerate}
	 \item Cambios de aplicaciones software.
	 \item Cambios de procedimientos.
	 \item Cambios de procesos.
	 \item Cambios de par�metros del sistema.
	 \item Cambios de par�metros de servicio.
	 \item Cambios de las plataformas fundamentales.
\end{enumerate}

Del mismo modo, los cambios realizados no son registrados en ning�n sitio ni previamente autorizados. Adem�s, no se garantizan los resultados de los cambios antes de realizarse
ya que no existe ning�n metodo de evaluaci�n de los impactos de los cambios en el sistema operacional y su funcionalidad. 

La empresa tambi�n carece de un plan de cambios de emergencia que defina, plantee, evalue y autorice los posibles cambios de emergencia.

Con respecto a los cambios aplicados no existe un sistema de seguimiento y reporte de los mismos que mantengan informados a los solicitantes del cambio y a los interesados
relevantes, acerca del estado de cambio de las aplicaciones, procedimientos, procesos, par�metros del sistema y del servicio y de las plataformas fundamentales.

En consecuencia con todo lo anterior, se carece de un proceso de revisi�n que garantice la implantaci�n corrrecta y completa de los cambios.


\end{spacing}
%\clearpage
%% -*- coding: utf-8 -*-

\subsection{Problemas relacionados con Administrar el Desempe�o y la Capacidad}

\begin{spacing}{1.4}

\end{spacing}
%\clearpage
%% -*- coding: utf-8 -*-

\subsection{Problemas relacionados con la continudidad del servicio}

\begin{spacing}{1.4}

%Marco de trabajo de continuidad de TI
La empresa carece de un marco de trabajo de continuidad de TI que de soporte a la continuidad del negocio. Esto es, no se tiene un plan de recuperaci�n de desastres y
de contingencias.

%Planes de continuidad
La empresa por tanto, no tiene definido un plan de continuidad de TI por lo que en caso de interrupci�n de los sistemas el impacto ser� mayor y en consecuencia se ver�n afectados
los procesos clave de negocio.

%Recuperaci�n y Reanudaci�n
De lo anterior se extrae que la empresa tendr�a dificultades para volver a su actividad normal en caso de un desastre o contingencia lo que podr�a provocar perdidas muy grandes para la 
empresa.

%Respaldo
Las copias de seguridad que posee la empresa no son suficientemente frecuentes. Adem�s carece de copias de seguridad fuera de las instalaciones por lo que en caso de desastre en las instalaciones
existe el riesgo de perder toda la informaci�n relevante de la empresa.


Podr�amos decir que la empresa se encuentra en un nivel de madurez Repetible pero Intuitivo. Se reconoce que los cambios se deben administrar y controlar pero no toman medidas.



\end{spacing}
%\clearpage
%Garantizar la seguridad de los sitemas

La seguridad de los sistemas se ha auditado desde el punto de vista del personal, 
del software y del hardware, y ser�n detallados a continuaci�n.

Para empezar, las licencias del antivirus y firewall utilizados por la empresa est�n 
caducadas no siendo posible actualizar las bases de datos de dichas aplicaciones, y 
las actualizaciones cr�ticas del sistema operativo empleado, Windows XP Professional, 
no han sido instaladas (ver Anexo ??). Esto expone a los computadores de la empresa 
a numerosos ataques realizados por hackers.

En cuanto a la administraci�n y gesti�n de cuentas de usuario, no 
existe una pol�tica que obligue a los usuarios del sistema a cambiar su 
contrase�a personal en un periodo definido de tiempo, no pudiendo ser �sta la 
misma que una contrase�a anterior. Esto hace que si un usuario no autorizado 
consigue acceso mediante alguna cuenta, podr� hacerlo de por vida ya que la 
contrase�a no ser� modificada en un futuro pr�ximo.

Adem�s, el sistema no proporciona un m�todo para que los usuarios puedan 
modificar su contrase�a, por lo que �stos est�n enviando correos electr�nicos 
al administrador notificando su login y el nuevo password que desean, no est�ndo 
�ste sujeto a una pol�tica que defina a qu� restricciones est� sometido (longitud 
m�nima de caracteres, uso de d�gitos, etc). Esto acarrea serios problemas ya que 
la contrase�a de cada usuario no es conocida por una �nica persona y dicho correo 
electr�nico puede ser interceptado por alg�n tercero indeseado. No terminando aqu�, 
las contrase�as se encuentran almacenadas en la base de datos en texto plano, 
de modo que cualquier persona que tenga acceso a �sta (p.e. el administrador de 
bases de datos) puede conocer la contrase�a de cualquier usuario.

Tambi�n existe una falta en cuanto la recolecci�n de cuentas basura. Actualmente 
el plan de seguridad define que el d�a 2 de Enero de cada a�o se proceder� a 
eliminar del sistema todas las cuentas de aquellos usuarios que no permanezcan en 
la empresa. Esto supone un serio problema ya que examinando los logs de accesos 
al sistema se ha detectado que un 3% �stos son de usuarios que ya no pertenecen 
a la plantilla de la empresa. Se advierte de que la informaci�n que dichos usuarios 
puedan extraer de la empresa podr�a ser vendida a la competencia. Adem�s, 
dichos logs corresponden �nicamente a los �ltimos 3 meses de actividad (ver Anexo 
?? -logs-), siendo 2 a�os lo que exige la ley.

A nivel f�sico, se han encontrado importantes vol�menes de informaci�n de 
car�cter privado y que no est�n debidamente protegidos y restringidos �nicamente 
a personal autorizado (ver Anexo ?? -foto de documentos o servidores fuera de 
un armario sin llave-), incluso partes de estos vol�menes desechados en los 
contenedores de la empresa sin haber sido destruidos previamente (ver Anexo ?? 
-foto de curriculos y cds sin destruir en un contenedor-).

Tampoco se han encontrado existencias de un libro de incidencias, tal y como 
dicta el art�culo 10 del Real Decreto 994/99, en que hagan constar los 
siguientes puntos:
a) el tipo de incidencia
b) El momento en que se produce
c) La persona que realiza la notificaci�n
d) A qui�n se le comunica
e) Los efectos que se deriven de la incidencia.

En el contexto del personal se ha detectado que algunos empleados se autentican 
en el sistema por medio de unas tarjetas con chip y que �stos se las dejan 
olvidadas cerca de su estaci�n de trabajo haciendo posible que cualquier 
persona pueda suplantar su identidad (ver Anexo ??).
%\clearpage
%Administrar la configuraci�n

La empresa dispone de una herramienta de soporte y un repositorio central para 
almacenar informaci�n relevante sobre los elementos de configuraci�n pero no se 
encuentra actualizado con las nuevas necesidades y adquisiciones de la empresa, 
es decir, inicialmente se grabaron todos los activos pero no se han ido grabando 
los cambios que �stos han sufrido. Un claro ejemplo de esta falta es la existencia 
de licencias caducadas de aplicaciones y que a�n se est�n instalando en nuevas 
computadoras. Este descontrol ha llevado a la empresa a re-adquirir 
licencias que ya pose�a y gastar un dinero innecesario (ver Anexo ??).

Por otro lado, no existe un prodecimiento que defina qu� software debe ser instalado 
en una m�quina cuando es introducida en la empresa. Esta labor la llevan a cabo 
dos t�cnicos que van instalando aplicaciones en los ordenadores conforme los usuarios 
las van necesitando y sin crear ning�n tipo de informe sobre los cambios que van 
haciendo a cada computadora.

Adem�s, se ha encontrado software personal y/o no licenciado (ver Anexo ?? -lista
de programas-) instalado en las computadoras de los empleados. Se le ha pedido 
al responsable en cuesti�n los informes de las revisiones pertinentes que se han 
llevado a cabo en dicho �mbito y no hemos obtenido respuesta alguna.


%\clearpage
%% -*- coding: utf-8 -*-

\subsection{Problemas relacionados con la administraci�n de los datos}

\begin{spacing}{1.4}
Continuando con la auditor�a, hemos observado que, con respecto a la web con la que la empresa realiza su expansi�n de negodio de venta por internet, los datos que se requieren a la hora de realizar una b�squeda son devueltos con bastante rapidez y exactitud, pero una gran deficiencia que cabe destacar es el hecho de que tanto los datos referidos a la informaci�n de los productos que se venden en la web como la informaci�n relevante de la empresa se almacenan en el mismo servidor.
Esto puede dar lugar a que datos de caracter personal sobre los empleados de la empresa pueden ser accedidos desde fuera de la empresa v�a web.
Cuando se pregunt� al personal a cargo del almacenamiento y seguridad de los datos sobre qu� procedimientos se segu�an para el correcto procesamiento y almacenamiento de los datos de la empresa se nos contest� que s� se hab�a definido un responsable para administrar esos datos y su almacenamiento, pero que no se le hab�a informado sobre la existencia de ning�n procedimiento, por lo que se hab�a optado por almacenar todos los datos en un mismo servidor, incluso en un mismo directorio dentro de �ste.
Otro fallo que encontramos, en �ste �mbito, dentro de la empresa fu� que no se hab�a adquirido un segundo servidor para replicar los datos y servicios relativos a la venta de art�culos por internet por lo que, en caso de un fallo en el primer y �nico servidor, el servicio de venta en la web quedar�a interrumpido durante horas o d�as...
Indagando un poco m�s se descubri� que s� se realizaban copias de seguridad de los datos, tanto los referidos a la informaci�n sensible a la empresa como la informaci�n de los producctos que venden; lamentablemente el soporte en el que se almacena la copia de respaldo de esos datos es un disco duro externo que se guarda en un caj�n, sin vigilancia ni medidas de seguridad, del mismo edificio en el que se encuentra el servidor.
A parte del encargado de administrar los datos, nadie accede, en principio, a los datos, pero no se toman las medidas de seguridad apropiadas para asegurarse de que no accedan a los datos personas no autorizadas.
A la hora de eliminar o destruir datos existen dos v�as para hacerlo: si los datos se encuentran en soporte electr�nico, es el administrador de datos el que los borra; y, si los datos est�n en papel, para eliminarlos se rompe el papel y se tira a la papelera. En cuanto al procedimiento a seguir cuando se estropea un soporte de almacenamiento, no est� especificado, por lo que simplemente, o bien se tira a la basura o bien se lleva a reciclar, pero no se destruye la informaci�n que contiene, por lo que algui�n podr�a recuperar esos datos de alguna manera usando un lector magn�tico...
\end{spacing}
%\clearpage
%% -*- coding: utf-8 -*-

\subsection{Problemas relacionados con el ambiente f�sico y su administraci�n}

\begin{spacing}{1.4}
Observando el entorno de trabajo y las instalaciones que posee la empresa vemos que a simple vista se carece de cualquier tipo de medida sobre la restricci�n o no del acceso a las instalaciones por parte del personal como por parte de personal ajeno a la empresa.
Cuando hemos pereguntado si exist�a y pod�amos revisar el plan de seguridad con las medidas relativas al acceso a las instalaciones se nos ha respondido negativamente. Es necesario detallar unas medidas de seguridad a tomar xa evitar el riesgo de que alguien acceda a equipos o informaci�n delicada...
Un punto d�bil que hemos encontrado relacionado con medidas de seguridad f�sica es la ausencia de un cortafuegos f�sico que impida el acceso al servidor desde el exterior de manera malintencionada pudiendo existir el riesgo de ataques que repercutan en la denegaci�n del servicio, etc.
Hemos observado que la elecci�n del centro de datos en el que tienen el servidor no se corresponde con el dise�o �ptimo que debe tener el espacio f�sico destinado a servidores, por lo que existe riesgo de que el servidor se caliente en exceso y se estropee dejando sin servicio de venta online a la empresa, lo que se traducir�a en cuantiosas p�rdidas econ�micas, tanto en ventas como en deterioro de la imagen de la empresa en internet.
En caso de incendio, inundaci�n o alg�n otro desastre, el servidor que contiene los datos podr�a estropearse, ya que, como hemos observado, el local en el que se ubica no es el apropiado ni est� dotado de medidas de prevenci�n y protecci�n ante tales eventualidades.
Finalmente descubrimos que no existe ning�n sistema de alimentaci�n ininterrumpida ni tampoco una fuente de corriente alternativa, por lo que cualquier corte de luz llevar�a forzosamente a una suspensi�n del servicio de venta por internet as� como del acceso a datos de otra �ndole, por no hablar de posibles da�os en el servidor y p�rdida de datos.
\end{spacing}
%\clearpage

\section{Papeles de Trabajo}
\phantomsection
\subsection*{ANEXO 1: Entrevista con el Director General de la empresa} 
\addcontentsline{toc}{subsection}{ANEXO 1: Entrevista con el Director General de la empresa}

\begin{spacing}{1.4}

\begin{milista}
	\item \textbf{Auditor:} En primer lugar, gracias por su colaboraci�n y por haberme facilitado los documentos que le solicit�. 
\end{milista}
\textit{De nada}.

\begin{milista}
	\item \textbf{Auditor:} �Cu�l es el cargo que desempe�a usted en la empresa?
\end{milista}
\textit{Soy el Director General de la empresa y pertenezco al Comit� de Direcci�n}.

\begin{milista}
	\item \textbf{Auditor:} �Cu�l es el principal objetivo de negocio de su empresa?
\end{milista}
\textit{Nuestro principal objetivo es mantenernos entre las 10 empresas m�s competitivas en la venta de componentes inform�ticos, tanto en tiendas como a trav�s de la Web}.

\begin{milista}
	\item \textbf{Auditor:} Para conseguir su objetivo de negocio, �cree que las tecnolog�as de informaci�n son algo fundamental que deben alinearse con los objetivos de la empresa?
\end{milista}
\textit{En realidad, nosotros utilizamos los recursos tecnol�gicos como una herramienta para conseguir nuestro objetivo y para que los distintos departamentos puedan llevar a cabo su trabajo, pero no integramos las tecnolog�as de la informaci�n en los planes de negocio}.

\begin{milista}
	\item \textbf{Auditor:} Por lo que me dice, �puede suponer que no cuentan con un plan estrat�gico de tecnolog�as de la informaci�n?
\end{milista}
\textit{As� es}.

\begin{milista}
	\item \textbf{Auditor:} Entonces, �qui�n es el responsable de las tecnolog�as de la informaci�n?
\end{milista}
\textit{Cada uno de los departamentos de la empresa cuentan con unas personas expertas en inform�tica que se encargan de todo lo relacionado con las tecnolog�as y los proyectos inform�ticos de la empresa: adquirir recursos, desarrollar proyectos, realizar mantenimiento, etc}.

\begin{milista}
	\item \textbf{Auditor:} Por tanto, tampoco cuentan con un plan de mantenimiento y adquisic�n de recursos tecnol�gicos. 
\end{milista}
\textit{As� es. Cuando los inform�ticos nos dicen que necesitan comprar nuevos recursos, nosotros aprobamos la adquisici�n}.

\begin{milista}
	\item \textbf{Auditor:} Si hay alg�n problema con los proveedores, �qu� medidas toman?
\end{milista}
\textit{No solemos tener problemas con los proveedores, porque llevamos varios a�os trabajando con ellos y tenemos buenas experiencias. Sin embargo, hubo un caso en el que el proveedor cancel� un pedido que ya estaba pagado, por lo que tuvimos que tomar medidas legales y comprar urgentemente esos recursos a otro proveedor. Finalmente, conseguimos recuperar el dinero, pero una secci�n del Departamento de Almacenaje y Facturaci�n estuvo paralizada durante unos d�as al no contar con esos recursos}.

\end{spacing}


\phantomsection
\subsection*{ANEXO 2: Situaci�n actual de las empresas competidoras}
\addcontentsline{toc}{subsection}{ANEXO 2: Situaci�n actual de las empresas competidoras}

Los gr�ficos de las Figuras \ref{fig:inversion} y \ref{fig:presencia} han sido obtenidos de ?????.

\begin{figure}[!ht]
\centering
\includegraphics[scale=0.7,keepaspectratio]{./images/investigacion}%
\caption{Porcentaje de inversi�n en investigaci�n de nuevas tecnolog�as}%
\label{fig:inversion}%
\end{figure}

\begin{figure}[!ht]
\centering
\includegraphics[scale=0.7,keepaspectratio]{./images/presencia}%
\caption{Presencia en el mercado de las empresas}%
\label{fig:presencia}%
\end{figure}

\phantomsection
\subsection*{ANEXO 3: Organigrama de la empresa DIGSOL} \label{anexo:organigrama}
\addcontentsline{toc}{subsection}{ANEXO 3: Organigrama de la empresa DIGSOL}

\begin{spacing}{1.4}
El organigrama mostrado en la Figura \ref{fig:organigrama} ha sido cedido por el Comit� de Direcci�n de la empresa DIGSOL.
\end{spacing}

\begin{figure}[h]
\centering
\includegraphics[scale=0.7,keepaspectratio]{./images/organigrama}%
\caption{Organigrama de la empresa DIGSOL}%
\label{fig:organigrama}%
\end{figure}

\phantomsection
\subsection*{ANEXO 4: Adquisiciones tecnol�gicas}
\addcontentsline{toc}{subsection}{ANEXO 4: Adquisiciones tecnol�gicas}

\begin{spacing}{1.4}

Los fragmentos de facturas mostradas en las Figuras \ref{fig:factura} y \ref{fig:facturaAntigua} han sido cedidas por la Direcci�n de la empresa DIGSOL. 

La Figura \ref{fig:facturaAntigua} muestra la adquisici�n que dicha empresa realiz� a mediados del a�o 2008, cuando implant� el servicio de venta Web y actualiz� los equipos de algunos de sus departamentos.

\begin{figure}[ht]
\includegraphics[scale=0.57,keepaspectratio]{./images/facturaAntigua}%
\caption{Fragmento de la factura del 17-Junio-2008}%
\label{fig:facturaAntigua}%
\end{figure}

En la factura de la Figura \ref{fig:factura} se muestra una de las �ltimas adquisiciones que ha realizado la empresa, remarcando gastos que podr�an haberse evitado (en color amarillo) y gastos totalmente innecesarios que no aportan nada de beneficio al negocio (en color rojo). 

\begin{figure}[ht]
\includegraphics[scale=0.57,keepaspectratio]{./images/factura}%
\caption{Factura del 27-Mayo-2009}%
\label{fig:factura}%
\end{figure}

%El servidor de respaldo es una adquisici�n necesaria, ya que la empresa no lo renovaba desde hace unos 3 o 4 a�os. Sin embargo, la compra de los 17 equipos de sobremesa, destinados al departamento de contabilidad de la sede auditada, es algo innecesario, ya que pr�cticamente la totalidad de equipos se renovaron hace poco m�s de un a�o (coincidiendo con la implantaci�n del sistema Web), tal y como muestra la factura de esa �poca (ver Figura \ref{fig:facturaAntigua}).

%Del mismo modo, adquirir un nuevo servidor para ampliar la capacidad de almacenamiento es, en el momento actual en el que se encuentra la empresa, algo innecsario, pues el servidor que se adquiri� en su dia tiene capacidad suficiente. Podr�a pensarse entonces en qu� es una adquisici�n �til para medio o largo plazo, pero esto no es as�, ya que las tecnolog�as avanzan r�pidamente y en un plazo relativamente corto, existir�n mejores soluciones en el mercado.

%En lo que se refiere a la compra de las PDAs para las personas que integran el comit� de Direcci�n, es un gasto totalmente no justificado e in�til, ya que no se utilizan dichas PDAs para proporcionar valor al negocio. Podr�an aprovecharse para sincronizar los datos de la empresa en las PDAs, atender negocios en ella, etc., pero esto no ocurre as�.
\end{spacing}

\phantomsection
\subsection*{ANEXO 5: Entrevista con uno de los desarrolladores inform�ticos de uno de los departamentos de la empresa}
\addcontentsline{toc}{subsection}{ANEXO 5: Entrevista con uno de los desarrolladores inform�ticos de uno de los departamentos de la empresa}

\begin{spacing}{1.4}

\begin{milista}
	\item \textbf{Auditor:} �Cu�l es el cargo que desempe�a usted en la empresa?
\end{milista}
\textit{Soy uno de los desarrolladores que formamos el equipo de desarrollo del departamento de Almacenaje y Facturaci�n}.

\begin{milista}
	\item \textbf{Auditor:} �Puede indicarme las responsabilidades que tiene asignadas?
\end{milista}
\textit{El equipo nos encargamos de desarrollar las aplicaciones que necesita este departamento para realizar su trabajo diario. Tambi�n somos responsables de adquirir nuevos recursos (tanto hardware como software) y de mantenerlos}.

\begin{milista}
	\item \textbf{Auditor:} �Qui�n aprueba las adquisiciones de tecnolog�a?
\end{milista}
\textit{Nosotros nos encargamos de elegir los recursos de entre diferentes proveedores y le pasamos el presupuesto a la Direcci�n}.

\begin{milista}
	\item \textbf{Auditor:} �Como realizan la lista de proveedores?
\end{milista}
\textit{Elegimos aquellos cuya relaci�n calidad/precio es mejor, adem�s de los proveedores con los que tenemos buenas impresiones personales}.

\begin{milista}
	\item \textbf{Auditor:} Si el proveedor no entrega los recursos en fecha o hay alg�n problema, �c�mo act�an?
\end{milista}
\textit{Nos dirijimos a la Direcci�n e intentamos seguir haciendo el trabajo con los recursos disponibles. Nosotros s�lo somos responsables de establecer los recursos que necesitamos y a qu� proveedor los compramos. Es la Direcci�n qui�n deber�a tomar medidas a la hora de realizar los pedidos}.

\begin{milista}
	\item \textbf{Auditor:} �Ha existido alg�n problema con alg�n pedido?
\end{milista}
\textit{En una ocasi�n uno de los proveedores cancel� un pedido sin dar explicaciones, por lo que tuvimos que realizar uno nuevo, pero una secci�n del departamento tuvo que estar paralizada unos d�as}.

\begin{milista}
	\item \textbf{Auditor:} Volviendo al tema de los proyectos, �tienen definido un portafolio de proyectos?
\end{milista}
\textit{Lo siento, pero, �qu� es un portafolio de proyectos?}.

\begin{milista}
	\item \textbf{Auditor:} Es una lista de proyectos a desarrollar, estableciendo sus prioridades.
\end{milista}
\textit{En realidad, vamos desarrollando los proyectos seg�n los empleados lo necesitan, pero no solemos asignar prioridades}.

\begin{milista}
	\item \textbf{Auditor:} �Y qu� sucede si hay un proyecto prioritario que la Direcci�n necesita para cumplir el objetivo de negocio de la empresa?
\end{milista}
\textit{En ese caso, solemos dividir el equipo de desarrollo, para que una parte atienda ese proyecto prioritario y el resto atienda los proyectos 
normales y el resto de tareas de las que nos encargamos}.

\begin{milista}
	\item \textbf{Auditor:} �Y tienen �xito siguiendo ese ''m�todo'' de trabajo?
\end{milista}
\textit{Muchas veces hemos tenido que echar bastantes horas extras, pero podemos entregar los proyectos en fecha. No podemos hacer otra cosa si la direcci�n no crea un Departamento especializado en estos temas y permite que nos limitemos �nicamente a desarrollar}.


\end{spacing}


\phantomsection
\subsection*{ANEXO 6: Instalaci�n y configuraci�n inicial del software en equipos inform�ticos}
\label{anexo:configuracion-inicial}
\addcontentsline{toc}{subsection}{ANEXO 6: Instalaci�n y configuraci�n inicial del software en equipos inform�ticos}

\begin{spacing}{1.4}
En la Figura \ref{fig:configuracion-inicial} se muestra la lista de software que ser� instalado en las computadoras. Como se puede apreciar, el documento es muy antiguo y no ha sido actualizado: las versiones del software est�n obsoletas, no se establece un criterio para aplicar parches y actualizaciones de seguridad, y no se establece una configuraci�n de par�metros y servicios del sistema.
\end{spacing}

\begin{figure}[h]
\centering
\includegraphics[scale=0.8,keepaspectratio]{./images/anexo_lista_config_inicial}%
\caption{Lista de la configuraci�n inicial de los equipos inform�ticos}%
\label{fig:configuracion-inicial}%
\end{figure}




\phantomsection
\subsection*{ANEXO 7: Vulnerabilidad a ataques inform�ticos} \label{anexo:ddos}
\addcontentsline{toc}{subsection}{ANEXO 7: Vulnerabilidad a ataques inform�ticos}

\begin{spacing}{1.4}
En la Figura \ref{fig:ataques-ddos} se muestra una gr�fica que representa los crecientes ataques inform�ticos que se producen sobre la organizaci�n. Esto se puede deber a la falta de mantenimiento y seguimiento de la seguridad de los sistemas de informaci�n.
\end{spacing}

\begin{figure}[h]
\centering
\includegraphics[scale=0.5,keepaspectratio]{./images/anexo_ataques_ddos}%
\caption{N�mero de ataques de Denegaci�n del Servicio durante 12 meses}%
\label{fig:ataques-ddos}%
\end{figure}



\phantomsection
\subsection*{ANEXO 8: Adquisici�n de licencias por duplicado} \label{anexo:repositorio-obsoleto}
\addcontentsline{toc}{subsection}{ANEXO 8: Adquisici�n de licencias por duplicado}

\begin{spacing}{1.4}
En las Figuras \ref{fig:avast1} y \ref{fig:avast2} se muestran dos facturas, proporcionadas por el Director de finanzas, pertenecientes a los meses de Enero y Junio del 2009 respectivamente, en las cuales se adquieren 30 licencias de antivirus de 1 a�o de duraci�n en cada una, haciendo un total de 60. Actualmente, la organizaci�n cuenta con 30 ordenadores por lo que el segundo pedido, que suma 676 euros, no estaba justificado. Esto se debe a que el repositorio central de licencias no fue actualizado cuando se hizo el primer pedido.
\end{spacing}

\begin{figure}[h]
\centering
\includegraphics[scale=0.6,keepaspectratio]{./images/anexo_factura_avast1}%
\caption{Factura de adquisici�n de licencias de antivirus en Enero del 2009}%
\label{fig:avast1}%
\end{figure}

\begin{figure}[h]
\centering
\includegraphics[scale=0.6,keepaspectratio]{./images/anexo_factura_avast2}%
\caption{Factura de adquisici�n de licencias de antivirus en Junio del 2009}%
\label{fig:avast2}%
\end{figure}



\phantomsection
\subsection*{ANEXO 9: Software con licencias caducadas}
\label{anexo:licencias-caducadas}
\addcontentsline{toc}{subsection}{ANEXO 9: Software con licencias caducadas}

\begin{spacing}{1.4}
En la Figura \ref{fig:licencias-caducadas} se puede observar que, a pesar de haber adquirido licencias para el a�o 2009-2010 (ver Figura \ref{fig:avast1}), a�n se mantienen licencias caducadas en los equipos.
\end{spacing}

\begin{figure}[h]
\centering
\includegraphics[scale=0.8,keepaspectratio]{./images/anexo_licencias_caducadas}%
\caption{Captura de pantalla del estado de un antivirus}%
\label{fig:licencias-caducadas}%
\end{figure}




\phantomsection
\subsection*{ANEXO 10: Software no corporativo}
\label{anexo:software-no-corporativo}
\addcontentsline{toc}{subsection}{ANEXO 10: Software no corporativo}

\begin{spacing}{1.4}
En la Figura \ref{fig:software-no-corporativo} se muestra un caso en el que un empleado tiene instalado en su ordenador un juego de Poker Online. El software personal y no alineado con el negocio de la empresa puede introducir fallas de seguridad.
\end{spacing}

\begin{figure}[h]
\centering
\includegraphics[scale=0.6,keepaspectratio]{./images/anexo_software_personal}%
\caption{Software no alineado con la estrategia de negocio}%
\label{fig:software-no-corporativo}%
\end{figure}



\phantomsection
\subsection*{ANEXO 11: Ubicaci�n y permisos de los archivos de datos} 
\addcontentsline{toc}{subsection}{ANEXO 11: Ubicaci�n y permisos de los archivos de datos}

\begin{spacing}{1.4}
En la captura que se muestra en la figura \ref{fig:archivos} se observa que los archivos de datos con informaci�n sensible, relativa a la empresa, y los archivos de la web con datos de los productos que se venden, se almacenan en el mismo servidor y en la misma cuenta y carpeta.
Adem�s los permisos de los archivos son inadecuados para ficheros que contienen informaci�n delicada.
\end{spacing}

\begin{figure}[h]
\centering
\includegraphics[scale=0.6,keepaspectratio]{./images/archivos}%
\caption{Ubicaci�n y permisos de archivos inadecuados}%
\label{fig:archivos}%
\end{figure}




\phantomsection
\subsection*{ANEXO 12: Cuentas de usuario} 
\addcontentsline{toc}{subsection}{ANEXO 12: Cuentas de usuario}

\begin{spacing}{1.4}
En la captura que se muestra en la figura \ref{fig:usuarios} se observan las cuentas de usuario que existen en el sistema. Esisten riesgos debido a que s�lo existen dos cuentas que son compartidas por todos los empleados de la empresa. Tambi�n hay riesgo por pertenecer las dos cuentas al mismo grupo de usuarios.
\end{spacing}

\begin{figure}[h]
\centering
\includegraphics[scale=0.6,keepaspectratio]{./images/usuarios}%
\caption{Usuarios del sistema}%
\label{fig:usuarios}%
\end{figure}




\phantomsection
\subsection*{ANEXO 13: Seguridad de la red wifi} 
\addcontentsline{toc}{subsection}{ANEXO 13: Seguridad de la red wifi}

\begin{spacing}{1.4}
En la captura que se muestra en la figura \ref{fig:aircrack} se ve como con el programa Aircrack se puede desencriptar la red inal�mbrica, por lo que la elecci�n de un buen sistema de cifrado es muy importante para la seguridad en la red si no se puede prescindir de la conexi�n wifi.
\end{spacing}

\begin{figure}[h]
\centering
\includegraphics[scale=0.6,keepaspectratio]{./images/usuarios}%
\caption{Programa que desencripta la wifi}%
\label{fig:aircrack}%
\end{figure}




\phantomsection
\subsection*{ANEXO 14: Vulnerabilidad del servidor} 
\addcontentsline{toc}{subsection}{ANEXO 14: Vulnerabilidad del servidor}

\begin{spacing}{1.4}
En la captura que se muestra en la figura \ref{fig:llaves_puestas} se ve como el servidor carece de las medidas de seguridad necesarias, adem�s de tener una ubicaci�n deficiente.
\end{spacing}

\begin{figure}[h]
\centering
\includegraphics[scale=0.6,keepaspectratio]{./images/llaves_puestas}%
\caption{Servidor vulnerable}%
\label{fig:llaves_puestas}%
\end{figure}



\phantomsection
\subsection*{ANEXO 15: Copias de seguridad e informaci�n accesibles} 
\addcontentsline{toc}{subsection}{ANEXO 15: Copias de seguridad e informaci�n accesibles}

\begin{spacing}{1.4}
En la captura que se muestra en la figura \ref{fig:cajones_abiertos} se ven los cajones en los que se guarda el disco con los datos de respaldo, pero no se cierran con llave.
\end{spacing}

\begin{figure}[h]
\centering
\includegraphics[scale=0.6,keepaspectratio]{./images/cajones_abiertos}%
\caption{Cajones donde se guarda el disco de respaldo}%
\label{fig:cajones_abiertos}%
\end{figure}




\phantomsection
\subsection*{ANEXO 16: Ficheros de log} 
\addcontentsline{toc}{subsection}{ANEXO 16: Ficheros de log}

\begin{spacing}{1.4}
En la captura que se muestra en la figura \ref{fig:logs} se ven los ficheeros de log, poco detallados y con poca informaci�n.
\end{spacing}

\begin{figure}[h]
\centering
\includegraphics[scale=0.6,keepaspectratio]{./images/logs}%
\caption{Ficheros de log}%
\label{fig:logs}%
\end{figure}



\phantomsection
\subsection*{ANEXO 17: Estanter�as desprotegidas} 
\addcontentsline{toc}{subsection}{ANEXO 17: Estanter�as desprotegidas}

\begin{spacing}{1.4}
En la captura que se muestra en la figura \ref{fig:papeles_inseguros} se ven las estanter�as donde se almacenan los documentos de la empresa, pero estas carecen de medidas de seguridad.
\end{spacing}

\begin{figure}[h]
\centering
\includegraphics[scale=0.6,keepaspectratio]{./images/papeles_inseguros}%
\caption{Estanter�as con documentos}%
\label{fig:papeles_inseguros}%
\end{figure}




\phantomsection
\subsection*{ANEXO 18: Ordenadores abiertos} 
\addcontentsline{toc}{subsection}{ANEXO 18: Ordenadores abiertos}

\begin{spacing}{1.4}
En la captura que se muestra en la figura \ref{fig:PC_abierto} se ve que algunos empleados abren las tapas de sus ordenadores, lo que provoca que estos puedan ser vulnerados.
\end{spacing}

\begin{figure}[h]
\centering
\includegraphics[scale=0.6,keepaspectratio]{./images/PC_abierto}%
\caption{Ordenador sin tapas}%
\label{fig:PC_abierto}%
\end{figure}




\phantomsection
\subsection*{ANEXO 19: Soportes de datos mal destruidos} 
\addcontentsline{toc}{subsection}{ANEXO 19: Soportes de datos mal destruidos}

\begin{spacing}{1.4}
En la captura que se muestra en la figura \ref{fig:disco_sin_destruir} se ve en la papelera un disco duro. Contiene informaci�n en su interior y es accesible facilmente.
\end{spacing}

\begin{figure}[h]
\centering
\includegraphics[scale=0.6,keepaspectratio]{./images/disco_sin_destruir}%
\caption{Disco duro destruido incorrectamente}%
\label{fig:disco_sin_destruir}%
\end{figure}




\phantomsection
\subsection*{ANEXO 20: Papeles con datos mal destruidos} 
\addcontentsline{toc}{subsection}{ANEXO 20: Papeles con datos mal destruidos}

\begin{spacing}{1.4}
En la captura que se muestra en la figura \ref{fig:papeles_mal_eliminados} se ve en la papelera un conjunto de documentos eliminados de forma incorrecta. Contienen informaci�n y es accesible facilmente.
\end{spacing}

\begin{figure}[h]
\centering
\includegraphics[scale=0.6,keepaspectratio]{./images/papeles_mal_eliminados}%
\caption{Papeles destruidos incorrectamente}%
\label{fig:papeles_mal_eliminados}%
\end{figure}




\phantomsection
\subsection*{ANEXO 21: Ordenadores sin bloquear} 
\addcontentsline{toc}{subsection}{ANEXO 21: Ordenadores sin bloquear}

\begin{spacing}{1.4}
En la captura que se muestra en la figura \ref{fig:ordenadores_accesibles} se ve un ordenador que no ha sido bloqueado cuando el empleado ha abandonado su puesto de trabajo.
\end{spacing}

\begin{figure}[h]
\centering
\includegraphics[scale=0.6,keepaspectratio]{./images/ordenadores_accesibles}%
\caption{Ordenador sin bloquear}%
\label{fig:ordenadores_accesibles}%
\end{figure}

\end{document}
