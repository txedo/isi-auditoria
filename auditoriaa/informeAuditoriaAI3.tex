% -*- coding: utf-8 -*-

\subsection{Problemas relacionados con Adquirir y Mantener Infraestructura Tecnol�gica}

\begin{spacing}{1.4}

% Relacionado con el punto AI3.1
Como ya se ha comentado, la empresa carece de un Departamento de Sistemas de Informaci�n que pueda, entre otras cosas, elaborar un plan para la adquisici�n y mantenimiento de recursos tecnol�gicos. Por ello, como la empresa carece de dicho plan, cuando cada departamento solicita nuevas tecnolog�as, la direcci�n lo aprueba. Por esta raz�n, la inversi�n en tecnolog�a que realiza la empresa es m�s elevada de lo necesario, lo que puede traducirse en p�rdidas econ�micas importantes a largo plazo. En el Anexo ?? (p�gina \pageref{anexo:facturas}) puede verse un ejemplo de gastos innecesarios en tecnolog�as. 

% Relacionado con el punto AI3.2
En lo que se refiere a la protecci�n de los recursos de infraestructura, realizando una visita por las instalaciones de la sede, se ha comprobado que la seguridad es m�nima. En primer lugar, los equipos de cada uno de los departamentos carecen de medidas antirrobo y se pueden manipular de una manera sencilla para sustraer partes importantes, como un disco duro (ver anexo ??) % Se puede hacer una foto de ordenadores abiertos y sin candados
Adem�s, el cableado de red de los diferentes equipos es visible, por lo que su desconexi�n y manipulaci�n es simple (ver Anexo ??) % Foto de cables

Por otra parte, los servidores residen en una sala a la que pueden acceder los desarrolladores inform�ticos de los distintos departamentos, lo que pone en riesgo la seguridad, integridad y disponibilidad de los servidores, ya que cualquiera de ellos puede manipularlos sin el consentimiento ni el conocimiento del resto. Adem�s, los servidores no cuentan con SAIs (Sistemas de Alimentaci�n Ininterrumpida), por lo que, frente a un fallo el�ctrico, la empresa quedar�a paralizada. % Foto de un armario de servidores sin proteccion ni SAI

Todo esto compromete la disponibilidad y seguridad de la infraestructura hardware, ya que la manipulaci�n de algunos recursos puede hacerse de una manera simple y sin controles internos que lo eviten.

Atendiendo a otros temas, tampoco se realiza un seguimiento y evaluaci�n de los recursos adquiridos, por lo que realmente la empresa no conoce si la inversi�n que realiza en tecnolog�a est� otorgando beneficio y s� realmente las nuevas tecnolog�as ayudan a alcanzar el objetivo de negocio de la empresa. 

% Relacionado con el punto AI3.3
Al igual que no existe un plan para la adquisici�n de infraestructura tecnol�gica, tampoco existe un plan para su mantenimiento. Simplemente, los encargados de inform�tica de cada departamento realizan las labores de mantenimiento necesarias cuando ocurre alg�n problema, al igual que se encargan de aplicar las actualizaciones necesarias al software instalado. Esto conlleva un descontrol, ya que no se relizan documentos acerca del mantenimiento realizado ni se sigue un control de cambios en el software que utiliza la empresa.


\end{spacing}