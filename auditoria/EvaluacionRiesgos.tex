\subsection {Evaluaci�n de riesgos, plan de contingencia y plan de continuidad}

\begin{spacing}{1.4}

La evaluaci�n y gesti�n de riesgos es un tema que no est� abordado de
una forma eficiente en la organizaci�n.

Aunque la empresa dispone de un plan de riesgos, �ste ha sido elaborado por algunos de
los miembros del Consejo de Administraci�n y no existe ning�n informe
en el cual se indique qui�nes de estos miembros estaban presentes en la
elaboraci�n del plan ni las personas a las que consultaron a la hora
de realizarlo. 

Adem�s, tampoco existe ning�n registro de la t�cnica de
identificaci�n de riesgos que emplearon para seleccionarlos, as� como el 
c�lculo de la probabilidad de ocurrencia y el impacto, lo cual puede 
significar que la elecci�n de �stos par�metros fuese arbitraria o manipulada. 
En el \textbf{Anexo 22} se muestra el acta de los asistentes a dicha reuni�n, cedida por el Comit� de Direcci�n.

La empresa por tanto, no tiene definido un plan de continuidad de TI por lo que en caso de interrupci�n de los sistemas el impacto ser� mayor y en consecuencia se ver�n afectados los procesos clave de negocio.

De lo anterior se extrae que la empresa tendr�a dificultades para volver a su actividad normal en caso de un desastre o contingencia lo que podr�a provocar perdidas muy grandes para la 
empresa.

Las copias de seguridad que posee la empresa no son suficientemente frecuentes. Adem�s carece de copias de seguridad fuera de las instalaciones por lo que en caso de desastre en las instalaciones
existe el riesgo de perder toda la informaci�n relevante de la empresa. En el \textbf{Anexo 23} se muestra c�mo se almacenan algunas de las copias de seguridad.

\end{spacing}