\subsection{Problemas relacionados con la Direción Tecnológica}

Para estudiar y decidir la direción tecnológica de la
empresa es necesario analizar las tecnologías existentes y emergentes,
de modo que se pueda planear y materializar la estrategia de TI. Dicho
esto, una vez más nos encontramos con los problemas que acarrean no
tener un departamento de I+D y un departamento de TI.

Ello conyeva, en el caso de la carencia del departamento de TI, no
tener personal especializado en la empresa con capacidad de
análisis y síntesis en cuanto a las tecnologías ya existentes en la
empresa, por lo que cuando surge una necesidad no se hace un estudio
previo para comprobar si es posible satisfacerla con los recursos
disponibles y símplemente se adquieren nuevos recursos o directamente
no se detecta dicha necesidad. Es más, a la hora de adquirir nuevos
recursos tecnológicos no se realiza un estudio previo de las
necesidades que éstos pueden satisfacer y simplemente se adquiere lo
más caro o lo que más sobrenombre tiene.

En cuanto a la carencia del departamento I+D, es dificil cumplir la
meta del Plan Estratégico ``mantener la empresa entre las más
competitivas del sector'' ya que no se tiene el potencia suficiente
para crear y aprovechar las nuevas oportunidades de negocio que nos
puede proporcionar dicho departamento.

Además, se ha detectado un gravísimo problema que hace que el sistema
de información de la empresa esté estancado y quede anticuado. Esto se
debe a que la empresa que lo desarrolló desapareció en el año 2001
dado que fue víctima del atentado de las Torres Gemelas y su única
sede y centros de datos se ubicaban en las mismas, existiendo en
agravante de que dicha empresa no proporcionó a DIGSOL la
documentación del desarrollo del sistema relativa al análisis y diseño
del mismo. Esto hace que incorporar nuevas funcionalidades al sistema
o migrar el sistema a una nueva presentación o sistema gestor de bases
de datos sea prácticamente imposible.