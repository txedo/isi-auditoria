% -*- coding: utf-8 -*-

\subsection{Problemas relacionados con la Planificaci�n Estrat�gica de TI}

\begin{spacing}{1.4}

% Problemas relacionados con el objetivo de control PO1.1, PO1.2,
% PO1.4

Consultando al Director General sobre la existencia de un plan estrat�gico de TI, se ha observado que no existe dicho plan. Por tanto, la empresa no est� alineada con las tecnolog�as de la informaci�n, lo que puede suponer, a largo o medio plazo, que la empresa comience a perder presencia en el mercado, ya que para 
seguir siendo competitiva, es necesario investigar nuevas tecnolog�as, actualizar las que ya tiene y aprovechar las oportunidades que ofrece las tecnolog�as de la informaci�n. En el Anexo I (p�gina \pageref{anexo:graficos}) se muestra la inversi�n que las empresas de la competencia realizan en I+D de nuevas tecnolog�as y la situaci�n actual en el mercado. Como se puede observar, si la empresa no alinea las tecnolog�as con su objetivo de negocio, en poco tiempo perder� la posici�n con respecto a su competidora m�s inmediata.

Derivado de la inexistencia de un plan estrat�gico de TI que alinee las tecnolog�as con el objetivo de negocio, aparece otro problema, que es la falta de un Departamento de Sistemas de Informaci�n en la organizaci�n de la empresa, tal y como se observa en el Anexo II (p�gina \pageref{anexo:organigrama}).
Esto provoca que, cuando cada departamento solicita nuevas tecnolog�as, la direcci�n lo aprueba, ya que no hay ning�n especialista que pueda administrar el valor de las TI actuales. Por esta raz�n, la inversi�n en tecnolog�a que realiza la empresa es m�s elevada de lo necesario, lo que puede traducirse en p�rdidas econ�micas importantes a largo plazo. En el Anexo III (p�gina \pageref{anexo:facturas}) puede verse un ejemplo de gastos innecesarios en tecnolog�as.

% Problemas relacionados con el objetivo de control PO1.6

Por otra parte, como no existe un plan que coordine todo lo relacionado con las tecnolog�as de la informaci�n, no existen portafolios de proyectos que asigne correctamente las prioridades a los diferentes proyectos que maneja la empresa. As�, los desarrolladores inform�ticos (que est�n asociados a los distintos departamentos de la empresa) se limitan a desarrollar las aplicaciones que dichos departamentos necesiten para llevar a cabo su tarea, en el orden en el que se van solicitando. Esto provocar� que no se atiendan a tiempo los proyectos m�s prioritarios para mantener el objetivo de negocio de la empresa y se produzcan p�rdidas econ�micas, adem�s de perder competitividad en el mercado.

% Como papel de trabajo, nos podemos inventar unos planes de proyecto 
% o unos calendarios donde se muestre que no se sigue un orden a la 
% hora de realizar los proyectos

\end{spacing}



