\subsection{Problemas relacionados con la Planificaci�n Estrat�gica de TI}

\begin{spacing}{1.4}

El objetivo a largo plazo de la empresa DIGSOL, definido en su plan estrat�gico, es mantenerse entre las empresas m�s competitivas del mercado.

Sin embargo, estudiando el plan estrat�gico cedido por el Director General de la empresa, as� como el organigrama, no se han establecido los procesos necesarios para cumplir dicho objetivo, ya que la empresa no cuenta con un departamento de I+D ni con un departamento de Tecnolog�as de la Informaci�n. Por este motivo, a medio o largo plazo, el objetivo de negocio de la empresa corre un riesgo y puede suponer p�rdidas econ�micas importantes, ya que, para que la empresa pueda seguir siendo competitiva, debe investigar nuevas tecnolog�as y actualizar las que ya tiene. De lo contrario, el actual sistema gestor de base de datos, as� como el resto del sistema, se quedar�a obsoleto y el resto de empresas ganar�an mercado  al satisfacer las necesidades de los clientes de una manera m�s rapida y eficiente.

Otro problema, relacionado con la inexistencia del departamento de Tecnolog�as de la Informaci�n, es que cada departamento solicita nuevas tecnolog�as frecuentemente, ya que no existe un director experto en Tecnolog�as de Informaci�n que pueda controlar a los distintos departamentos y sacar el m�ximo provecho a las tecnolog�as existentes. Esto provoca que la inversi�n en tecnolog�a sea m�s elevada de lo necesario, ya que la Direcci�n aprueba las peticiones de los distintos departamentos.

Como ya se ha dicho, siguiendo el organigrama de la empresa, �sta carece de un departamento de Tecnolog�as de la Informaci�n, por lo que los desarrolladores inform�ticos est�n asociados a los distintos departamentos de la empresa y su trabajo consiste en desarrollar las aplicaciones que dichos departamentos necesiten para llevar a cabo su tarea. Esto supone un problema en cuanto a las prioridades del negocio, ya que los proyectos se van desarrollando en el orden en que son propuestos por los distintos departamentos (tras la aprobaci�n de la Direcci�n). Esto puede provocar que se desarrollen en primer lugar proyectos de menos importancia en la empresa y que se demoren los proyectos m�s prioritarios, lo que puede suponer, a medio o largo plazo, p�rdidas econ�micas para la empresa.

\end{spacing}



