% -*- coding: utf-8 -*-

\subsection{Problemas relacionados con la Arquitectura de la Informaci�n}

\begin{spacing}{1.4}

%En cuanto a diccionario de datos empresarial y reglas de sintaxis

La empresa DIGSOL carece de un diccionario de datos empresarial que defina las reglas
de sintaxis para los datos de la organizaci�n lo que produce:
-Peor entendimiento entre los usuarios del sistema de TI y del Negocio.
-Posibilidad de creaci�n de elementos de datos incompatibles.

%En cuanto al esquema de clasificacion de datos y la administraci�n de la integridad
Existe un esquema de clasificaci�n de datos definido en el que cada tipo de usuario tiene
acceso a una informaci�n concreta pero al no encriptar la informaci�n que viaja en texto plano
por la red local es f�cilmente interceptable por los dem�s usuarios de la empresa incluso por usuarios
externos ya que adem�s la protecci�n de la wifi se realiza mediante clave WEP que es un tipo de clave 
f�cil de obtener mediante aplicaciones como AirCrack lo que compromete la integridad del sistema.

Los USB y los dispositivos �pticos de los ordenadores de la empresa no 
est�n bloqueados, lo que permitir�a que un usuario malintencionado tomar documentos
 y datos de car�cter confidencial para la empresa, introducir virus o troyanos en el sistema.
%En cuanto a Administraci�n de integridad

No existen procedimientos peri�dicos para cambio de contrase�as de los
usuarios en el sistema. Estas claves son facilitadas en la incorporaci�n del
usuario a la empresa y no son eliminadas tras su despido por el departamento de
SI cuando recibe la confirmaci�n pertinente.

El diagrama de la base de datos es correcto pero su implementaci�n no es coherente con el diagrama dise�ado.
Una vez hemos estudiado la base de datos hemos comprobado que las claves ajenas est�n ausente en la creaci�n 
de las tablas, as� como campos que no deben estar vacios est�n definidos con la posibilidad de serlo.

El acceso a la base de datos es posible si se conoce la cadena de conexi�n a la misma ya que carece de clave de
usuario y de acceso ya que se ha utilizado SQL Server con modo de autenticaci�n de Windows. Es posible obtener 
la ruta de acceso a la base de datos utilizando aplicaciones sniffer que se utilizan para escuchar la 
informaci�n que viaja por nuestra red local de forma que podemos obtener m�s informaci�n de la que el usuario
debiese obtener.

Carece de una base de datos centralizada que contenga la informaci�n completa de la empresa, hay una base de datos
que almacena una parte (productos,empleados,usuarios registrados...) y otra parte que se almacena en ficheros de texto
como facturas... Puesto que las facturas est�n relacionadas con productos, usuarios registrados y empleados entre otros
estas deber�an estar integradas en la base de datos centralizada.



\end{spacing}

