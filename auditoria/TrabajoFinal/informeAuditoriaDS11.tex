% -*- coding: utf-8 -*-

\subsection{Problemas relacionados con la administraci�n de los datos}

\begin{spacing}{1.4}
Continuando con la auditor�a, hemos observado que, con respecto a la web con la que la empresa realiza su expansi�n de negodio de venta por internet, los datos que se requieren a la hora de realizar una b�squeda son devueltos con bastante rapidez y exactitud, pero una gran deficiencia que cabe destacar es el hecho de que tanto los datos referidos a la informaci�n de los productos que se venden en la web como la informaci�n relevante de la empresa se almacenan en el mismo servidor.
Esto puede dar lugar a que datos de caracter personal sobre los empleados de la empresa pueden ser accedidos desde fuera de la empresa v�a web.
Cuando se pregunt� al personal a cargo del almacenamiento y seguridad de los datos sobre qu� procedimientos se segu�an para el correcto procesamiento y almacenamiento de los datos de la empresa se nos contest� que s� se hab�a definido un responsable para administrar esos datos y su almacenamiento, pero que no se le hab�a informado sobre la existencia de ning�n procedimiento, por lo que se hab�a optado por almacenar todos los datos en un mismo servidor, incluso en un mismo directorio dentro de �ste.
Otro fallo que encontramos, en �ste �mbito, dentro de la empresa fu� que no se hab�a adquirido un segundo servidor para replicar los datos y servicios relativos a la venta de art�culos por internet por lo que, en caso de un fallo en el primer y �nico servidor, el servicio de venta en la web quedar�a interrumpido durante horas o d�as...
Indagando un poco m�s se descubri� que s� se realizaban copias de seguridad de los datos, tanto los referidos a la informaci�n sensible a la empresa como la informaci�n de los producctos que venden; lamentablemente el soporte en el que se almacena la copia de respaldo de esos datos es un disco duro externo que se guarda en un caj�n, sin vigilancia ni medidas de seguridad, del mismo edificio en el que se encuentra el servidor.
A parte del encargado de administrar los datos, nadie accede, en principio, a los datos, pero no se toman las medidas de seguridad apropiadas para asegurarse de que no accedan a los datos personas no autorizadas.
A la hora de eliminar o destruir datos existen dos v�as para hacerlo: si los datos se encuentran en soporte electr�nico, es el administrador de datos el que los borra; y, si los datos est�n en papel, para eliminarlos se rompe el papel y se tira a la papelera. En cuanto al procedimiento a seguir cuando se estropea un soporte de almacenamiento, no est� especificado, por lo que simplemente, o bien se tira a la basura o bien se lleva a reciclar, pero no se destruye la informaci�n que contiene, por lo que algui�n podr�a recuperar esos datos de alguna manera usando un lector magn�tico...
\end{spacing}