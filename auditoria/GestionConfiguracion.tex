\subsection {Gestión de la configuración y control de cambios}

\begin{spacing}{1.4}

La empresa dispone de una herramienta de soporte y un repositorio central para almacenar información relevante sobre los elementos de configuración hardware y software. El objetivo es garantizar la integridad de las configuraciones, pero este repositorio no se encuentra actualizado con las nuevas adquisiciones y necesidades de la empresa, ni con la evolución de las TI.

En el Anexo ?? se observa un fragmento del documento que especifica la configuración inicial de los equipos que utilizarán los empleados. Como se puede apreciar, el software que inicialmente se instalará en estos equipos está obsoleto y no existe una norma estándarice la instalación de parches y actualizaciones de seguridad, así como la configuración de servicios y parámetros del sistema. Esto hace que los sistemas no sean seguros y puedan ser vulnerables a ataques informáticos (ver Anexo ??).

Otra de las consecuencias que acarrea no tener actualizado el repositorio afecta al uso y adquisición de licencias. En el Anexo ?? se puede apreciar que en el último año se han adquirido licencias duplicadas para el antivirus y que aún así existen equipos con licencias caducadas. Este problema, sólo en el caso del antivirus, ha supuesto un total de XXXXXX euros.

Por otro lado, los cambios realizados no son registrados en ningún documento ni previamente autorizados por el Director de Informática, y no se mantiene una línea base de los elementos de la configuración para todos los sistemas y servicios como punto de comprobación al que volver tras el cambio. Además, no se garantizan los resultados de los cambios antes de realizarse ya que éstos no están sujetos a ningún tipo de norma y cada usuario los hace libremente. Como prueba de ello, en el Anexo ?? se observan capturas de pantalla de software personal y/o no licenciado que ha sido instalado en los equipos, acción que podría haber sido corregida si los equipos se hubiesen revisado periódicamente por el Gerente de Configuración.

\end{spacing}