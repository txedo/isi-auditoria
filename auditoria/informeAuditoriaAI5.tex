% -*- coding: utf-8 -*-

\subsection{Problemas relacionados con la adquisici�n de recursos de TI}

\begin{spacing}{1.4}
Al realizar la auditor�a nos hemos fijado en estas �reas dentro de la compa��a: infraestructura hardware, software y la seguridad de los datos.
Respecto a la infraestructura hardware de la empresa, y m�s concretamente, el proceso de adquisici�n de hardware se ha detectado que ni existe ni est� definida una pol�tica a seguir a la hora de adquirir nuevos dispositivos hardware.
Este es, claramente, un riesgo para el TI ya que no se controla qui�n realiza la adquisici�n de hardware, ni si previamente alguien le ha dado una lista de dispositivos que se deben adquirir, etc. No se especif�ca en ning�n documento qui�n es el responsable del desarrollo de unos procedimientos a seguir a la hora de adquirir nuevos dispositivos, existe una lista de proveedores, pero no se ind�ca qui�n estableci� esalista ni qui�n debe actualizarla.
Tampoco se indica si se consulta o no al CEO y al CFO, por lo que existe el riesgo de que cualquier empleado cometa fraudes pudiendo establecer como proveedor a uno con el que haya pactado pero cuyos precios no se ajusten a la pol�tica de gastos o precios de la empresa.
Con respecto a la adquisici�n del software si existen unas pol�ticas definidas. Una vez revisadas se observan algunos riesgos como son la ausencia de un responsable claro que rinda cuentas sobre qu� lista de proveedores se ha hecho y c�mo se mantiene esta (acualiz�ndola a�adiendo o eliminando proveedores).
Una vez que se realiza un pedido, ya sea de hardware o de software, no existe un procedimiento por el que se sigan unas pautas a la hora de realizar pagos o devoluciones; tampoco se indican cl�usulas  u obligaciones que deben cumplirse por parte de los proveedores. Adem�s otro riesgo detectado es que ninguno de los contratos firmados con los proveedores ha sido revisado por un asesor legal.
Al no establecerse esas clausulas contractuales, si un proveedor no satisface un pedido en el periodo acordado puede repercutir en graves p�rdidas econ�micas, debido a que la empresa vende productos por internet y la no disponibilidad de un producto haga que se pierda su venta y que la credibilidad y seriedad de la empresa vea da�ada su imagen.

\end{spacing}